\chapter{Introduction} 
\chapterimage[width=15cm]{wordcloud/chap1b.png}
Neutrons stars are curious objects encompassing many still unsolved problems of modern physics and astrophysics.
Their unique nature makes them ideal laboratories for many of the most energetic phenomena in space.
Born from the ashes of a supernova, they begin their life only when some other normal star fades away and dies in a spectacular supernova explosion.%
\footnote{ For the pedantic ones, we specify that a supernova is actually an implosion followed by a subsequent explosion.}
From there on, they continue their life by slowly collecting the surrounding interstellar matter or by devouring an unlucky companion star floating next to them.
It is not the impressive $\sim 10^{30}$ kilograms they weight but the mere $\sim 10 \km$ in radius sphere that they encapsulate this material into that is then able to bend the spacetime itself.
Such an impressive feat rewards them with a categorization into a special stellar group called \textit{compact objects}.
Along with white dwarfs and black holes, these strange compact objects have been under a scientific scrutiny for almost nine decades now.
Still, some of the most fundamental questions remain:
What are neutron stars made of?
How big are they?
How do we see them?

\section{Short history}
\subsection{From imagination into a reality}

In 1908, Lev Landau was born in Baku, Azerbaijan.
Lev turned out to become a brilliant Soviet scientist who already at an age of 14 matriculated from the Baku State University studying in both, Department of Physics and Mathematics, and the Department of Chemistry simultaneously.
Soon after that he became disinterested in chemistry but continued to be fascinated by physics through his whole career.
In 1931, at an age of 23, Landau published an exceptional paper where he speculated on the existence of stars containing matter on nuclear densities:\cite{Landau32}
%In 1931, at an age of 23 years, a Soviet physicist Lev Landau published an exceptional paper where he speculated on the existence of stars containing matter on nuclear densities:\cite{Landau32}
\begin{displayquote}
    \textit{...the density of matter becomes so great that atomic nuclei come in close contact, forming one gigantic nucleus.}
\end{displayquote}
What made this paper exceptional\cite{Rutherford20} was that it was done before the existence of neutrons\cite{Landau32} were confirmed, hence proposing that quantum mechanics might be violated as the atoms consisting of protons and electrons only would live together.\cite{NSK16, Phillips94}
Nevertheless, it marked the first theoretical speculation on the existence of what we now know as neutron stars.\footnote{another footnote here}


\subsection{another test section}
In 1908, Lev Landau was born in Baku, Azerbaijan.
Lev turned out to become a brilliant Soviet scientist who already at an age of 14 matriculated from the Baku State University studying in both, Department of Physics and Mathematics, and the Department of Chemistry simultaneously.
Lev turned out to become a brilliant Soviet scientist who already at an age of 14 matriculated from the Baku State University studying in both, Department of Physics and Mathematics, and the Department of Chemistry simultaneously.
Lev turned out to become a brilliant Soviet scientist who already at an age of 14 matriculated from the Baku State University studying in both, Department of Physics and Mathematics, and the Department of Chemistry simultaneously.
Soon after that he became disinterested in chemistry but continued to be fascinated by physics through his whole career.\footnote{what about in this page?}
In 1931, at an age of 23, Landau published an exceptional paper where he speculated on the existence of stars containing matter on nuclear densities:\cite{Landau32}




