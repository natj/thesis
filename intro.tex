\chapter{Introduction} 
\chapterimage[width=15cm]{wordcloud/chap1b.png}

Neutron stars are curious objects encompassing many problems of modern physics and astrophysics still unsolved.
Their unique nature makes them ideal laboratories for many of the most energetic phenomena in space.
Born from the ashes of a supernova, they begin their life only when a normal star fades away and dies in a spectacular explosion.%
\footnote{ For the pedantics, we specify that a supernova is actually an implosion followed by a subsequent explosion.}
From there on, they continue their life by slowly collecting the surrounding interstellar matter or by devouring an unlucky companion star floating next to them.
It is not the $\osim 10^{30}$ kilograms they weigh but the sphere mere $\osim 10 \km$ in radius that they encapsulate this material into that is then able to bend spacetime itself.
Such an impressive feat awards them a categorization of a special stellar group called \textit{compact objects}.
Along with white dwarfs and black holes, these strange compact objects have been under scientific scrutiny for almost nine decades.
Still, some of the most fundamental questions remain:
What are neutron stars made of?
How big are they?

In the next few sections, we will discuss these peculiar objects habitating the Space around us in detail.
First, a short history of their discovery is given, followed by simple physical arguments on why we actually expect them to exist in the first place.
This will hopefully help us in building some intuition about physical phenomena possible near and inside these stars.


\section{Short history}
\subsection{From imagination to reality}

In 1908, Lev Landau was born in Baku, Azerbaijan.
Landau turned out to become a brilliant Soviet scientist, who already at the age of 14 matriculated at the Baku State University studying simultaneously in both the Department of Physics and Mathematics and the Department of Chemistry.
Soon after, he became disinterested in chemistry but continued to be fascinated by physics throughout his whole career.
In 1931, at the age of 23, Landau published an exceptional paper where he speculated on the existence of stars containing matter at nuclear densities:\cite{Landau32}
\begin{displayquote}
    \textit{...the density of matter becomes so great that atomic nuclei come in close contact, forming one gigantic nucleus.}
\end{displayquote}
What made this paper exceptional was that it was written before the existence of neutrons was confirmed. 
For contemporary science, this meant a violation of quantum mechanics, since atoms were thought to consist of protons and electrons only, and they certainly could not exist together in the same place inside such a hypothetical star.
Nevertheless, it marked the first theoretical speculation on the existence of what we now know as neutron stars.

Landau needed not wait for long.
Already next year, in 1932, James Chadwick confirmed that neutrons really were a fundamental part of our nuclear physics with his works dubbed \textit{Possible Existence of a Neutron}\cite{Chadwick32a} and the follow-up \textit{Existence of a Neutron}\cite{Chadwick32b}.
His experimental findings then confirmed the theoretical predictions his supervisor Ernest Rutherford made already in 1920\cite{Rutherford20}.
Later on,  in 1935, Chadwick was awarded the Nobel Prize in Physics for his findings.
Chadwick himself continued his career as part of the Manhattan project, as it was basically his groundbreaking work that inspired the U.S. government to begin serious research into the atomic bomb.


Now that the existence of the neutron was confirmed, it did not take long for others, independent of Landau, to propose similar stars.
At the Meeting of the American Physical Society at Standford in December 1933, one year after the neutron discovery, Wilhelm Baade and Frank Zwicky made their famous proposal that Supernovae should be considered a new category of astronomical objects.\cite{Baade34a, Baade34aa}
At the same time, they also stated:
\begin{displayquote}
    \textit{...we advance the view that a super-nova represents the transition of an ordinary star into a neutron star, consisting mainly of neutrons. Such a star may possess a very small radius and an extremely high density.}
\end{displayquote}
Such statements were, however, deemed a work of imagination by a bunch of weird astronomers.
Zwicky, on the other hand, kept on insisting that neutron stars really are out there.
Much later on, A.G.W. Cameron, a former post-doc at Caltech (where Zwicky was also situated) during 1959$-$1969, recalls:
\begin{displayquote}[A.G.W. Cameron, 1999]
    \textit{For years Fritz [Zwicky] had been pushing his ideas about neutron stars to anyone who would listen and had been universally ignored. 
    I believe that the part of the problem was his personality, which implied strongly that people were idiots if they did not believe in neutron stars.
    }
\end{displayquote}


Progress on the theoretical understanding of neutron stars was also tightly connected to understanding the interiors of white dwarfs.
Unlike the mysterious nuclear forces related to neutrons, the physics of white dwarfs was more related to understanding the behavior of electrons.
A breakthrough in this field came in 1925, when a young Paul Dirac formulated the quantum wave equations for the motion of electrons\cite{Dirac25}.
What soon followed was a description of the pressure of degenerate electron gas by Ralph Fowler, Dirac's supervisor\cite{Fowler26}.
The implications were severely against the previously known physics;
even in zero temperature, there would be a degeneracy pressure preventing matter from collapsing due to the exclusion principle of quantum mechanics.

Using a simplified uniform density approximation, Edmund Stoner was then able to show that this implied a maximum mass limit for white dwarfs.\cite{Stoner30}
Thus, a surprising result was obtained: when the density of a white dwarf approaches infinity, the mass reaches a maximum value.
The German-Estonian Wilhelm Anderson later realized that the electrons in this problem must actually be treated relativistically\cite{Anderson29}, something overlooked by Stoner.
Anderson tried to correct the crude mistake by deriving the equation of the state of relativistic degenerate electron gas but ended up making severe mistakes.
It was Stoner who corrected his equations based on communication with Anderson and re-derived his maximum mass limit.
Regardless of Stoner's efforts, it was later named Chandrasekhar's mass limit for its importance in astrophysics.

This was to honor Subrahmanyan Chandrasekhar, a young and prolific Indian physicist and astrophysicist, who was working on the same topic after reading Fowler's paper on degenerate electron gas.
Unlike Stoner's limit computed using the uniform density approximation, Chandrasekhar realized that a polytropic density profile is a more physical albeit mathematically more challenging formulation.
Still, the 19-year-old Chandrasekhar, already known for his mathematical prowess, was able to integrate the equations numerically by hand and obtained a similar limiting mass\cite{Cha31}.
Later on, however, it has been found that Chandrasekhar was not even the second person to derive the mass limit, but the third:\cite{Yakovlev94}
the Soviet physicist Yakov Frenkel published a similar derivation, independently and unknowingly of the progress in the west, in which he applied the relativistic degenerate electron gas results to white dwarfs and concluded that an upper limit on the mass must exist\cite{Frenkel28}.
However, his work went by unnoticed.


Nevertheless, the maximum mass for a white dwarf had been laid out, and in the end, after all the relevant physical inclusions, it turned out to be $1.44~\Msun$, or $1.44$ times the mass of our Sun.
What makes this limit important for us, is that the maximum mass for a white dwarf is related to the minimum mass for a neutron star,
an important connection first made by George Gamow in 1939\cite{Gamow39}.
The idea behind it is simple: 
if the degenerate electron gas pressure, quantum mechanical in nature, is what keeps the white dwarfs from collapsing, what happens when the maximum mass is reached and even this strange pressure is unable to resist the forces of gravity?
At a conference in Paris in 1939, Chandrasekhar laid out the answer:
\begin{displayquote}
    \textit{
    If the degenerate core attains sufficiently high densities, the protons and electrons will combine to form neutrons. 
    This would cause a sudden diminution of pressure resulting in the collapse of the star to a neutron core.
    }
\end{displayquote}
A neutron star should thus have a mass smaller than the Chandrasekhar limit, i.e. $M \sim 1.44\Msun$, and consist of neutrons only, exactly like proposed by Landau eight years earlier without the knowledge of neutrons, or later on by Baade and Zwicky when they presented their theory of supernovae!
%\red{JP: if WD would collapse to NS, the gravitational mass will decrease as NS has a significant binding energy. }


It was before the Second World War that a solid basis for a theory of neutron stars was established.
This was, however, just the beginning. 
The next question would be the critical one that we are still trying to answer today:
if they exist, how big are they?
The problem was that because of the extremely dense nature of these objects, the classical stellar equilibrium equations were no longer valid, and thus it was not possible to even estimate the size of a neutron star.
The problem was unwieldy due to its general relativistic nature;
the immense mass of the neutron star was bending spacetime itself, and the more compact it was, the more it could bend it. 
On the other hand, the more curved spacetime was, the more the star would gain weight and the more compact it would become.

It was already during the same year as Gamow's remark, in 1939, that a theoretical framework for studying this problem was published.
This was done independently by Richard Tolman\cite{Tolman39} and Robert Oppenheimer together with his student George Volkoff\cite{OV39}.
Both papers were even submitted on the same day, the 3rd of January, to Physical Review and were published on the same February issue.
More importantly, they both described a hydrostatic equilibrium for a spherically symmetric object in general relativity, exactly what was needed to study neutron stars.
Because of its great importance, the solution is now known as the Tolman-Oppenheimer-Volkoff equation.
In addition, Oppenheimer and Volkoff applied their equation and numerically calculated the structure of a neutron star consisting of non-interacting strongly degenerate neutron gas.
This marked the very first attempt in characterizing neutron stars.
Similar to white dwarfs, they also obtained an upper limit for their mass. 
However, as a disappointment for everyone, it was calculated to be around $0.7\Msun$, i.e. less than the Chandrasekhar limit of $1.44\Msun$ for white dwarfs, indicating that neutron stars could not exist in nature.
It took almost two more decades to show that it was actually the assumption of no interaction between the neutrons that was causing this hiccup.

Moreover, it was actually not Tolman nor Oppenheimer and Volkoff who first discovered the general relativistic hydrostatic equation.
It was now Chandrasekhar's turn to avoid having an important result credited to him;
together with John Von Neumann, Chandrasekhar extended his work on white dwarfs to also cover neutron stars and in the process derived exactly the same equilibrium equation in 1934, five years before the groundbreaking publication of Tolman, Oppenheimer and Volkoff.\cite{Baym82}
It is, however, worth mentioning that later on, in 1983, Chandrasekhar received the Nobel Prize in Physics for his work on ``theoretical studies of the physical processes of importance to the structure and evolution of the stars''.
So he certainly received at least some credit for his important work.

Around the same time, in 1937, Gamow and Landau also independently proposed that the accretion of matter onto a dense neutron star core could be the missing source of energy for stars.
This increased the interest towards neutron stars, and the field flourished in the 1930s.
However, it was soon shown that stars are powered not by accretion but by thermonuclear reactions as first suggested in the 1920s by Sir Arthur S. Eddington.\cite{Eddington26}
The interest in neutron stars then faded away and the research focused on weaponizing the nuclear forces.


The next big breakthrough came almost 20 years later in the 1950s, when John Wheeler and his collaborators constructed the first realistic equation of the state of dense matter\cite{Wheeler66}.
For the outer layers, known as the crust, they applied a semi-empirical mass formula together with the equation of the state of degenerate electrons.
For the dense core, they assumed a mixture of three ideal Fermi gases composed of neutrons, protons, and electrons.
This marked the first consistent formulation of neutron star structure.
It was followed by Cameron, who applied the Skyrme equation of state for the high-density matter.\cite{Cameron59}
This had important implications, as he was then able to show that the nuclear forces stiffen the matter considerably in comparison to the non-interacting free neutrons.
Similar to Tolman and Volkoff, he then went to calculate the maximum possible mass of a neutron star and arrived at approximately $2\Msun$.
This was an important theoretical breakthrough as it implied that neutron stars can, after all, exist.
A new wave of interest towards neutron stars was thus launched as everybody wanted to observe them.


\subsection{Many observational faces of neutron stars}

After Wheeler and Cameron had laid the modern foundation for studies on neutron star structure, everyone was eager to find these strange objects in the night sky.
It did not take long before researchers realized that as neutron stars are born in the supernova explosions, we expect them to be hot.
Most of the theoretical effort in the 60s was then focused on developing models for the cooling of neutron stars.\cite{Stabler60, Chiu64, Morton64, CS64, BW65a, BW65b, TC66}
It was the potential thermal radiation from this cooling that could then be used to detect them, as was first shown by Hong-Yee Chiu\cite{Chiu64}.
The first calculations predicted surface temperatures of $T \sim 10^6 \Kelvin$ for a neutron star of the age of around 1000 years.
This had important implications for the observers as it meant that neutron stars would mainly radiate in the range of X-rays.
The atmosphere of Earth, on the other hand, was impenetrable to the X-ray wavelengths.
Luckily, the 60s also marked the beginning of a golden era for spaceborn observatories.

Since X-rays could not reach the surface of the Earth, humankind went into space to observe them.
In the late 1950s and early 1960s, it was the pioneering experiments of the Italian-American astrophysicist Riccardo Giacconi that opened this new window into the Universe.
Giacconi first started with rocket-borne experiments and later continued by leading the development of the first orbiting X-ray satellite Uhuru, ``\textit{freedom}'' in Swahili.\cite{GGP62}
After the first X-ray satellite, Giacconi continued with the Einstein Observatory, the first fully imaging X-ray satellite, and later with the Chandra X-ray observatory.
For all of his efforts, he received the Nobel Prize in Physics in 2002 ``for pioneering contributions to astrophysics, which have led to the discovery of cosmic X-ray sources''.


During the starting boom, several extra-terrestrial X-ray sources were discovered.
As is common in science, the first discovery actually came by accident.
A team led by Giacconi launched an Aerobee 150 rocket to the skies in June 1962 with a payload of a highly sensitive soft X-ray detector meant to observe the X-rays from the Moon.
Due to a slight change (or a mistake) in the planned trajectory, it ended up observing the constellation of Scorpius and caught a glimpse of what is now known as the first extraterrestrial X-ray source, Sco X-1.
Little did they know that this was actually the first neutron star radiating towards us.
Five years later, in 1967, Iosif Shklovsky was the first to propose that Scorpius X-1 is a neutron star\cite{Shklovsky67}, but his work attracted little to no attention.

The first deliberate searches of neutron stars were aimed at the Crab Nebula, a well-known candidate for hosting a neutron star.
The Crab Nebula, already known in the 1920s and 1930s to be a supernova remnant is known to have exploded exactly on the 4th of July, 1054.\cite{Oort, Lundmark21, Mayall39, HistSupernovas}
In contemporary Chinese, Japanese, and Arab history writings, a ``guest star'' is described to appear in the constellation of Taurus and to persist even in broad daylight for 23 consecutive days.
Even after that, it remained visible in the night sky for two years.
For astronomers, this was a clear sign of a nearby supernova going off.


But it was not only the spectacular supernova but what was left behind that eluded astronomers.
Already in 1942, our old friends Baade and Rudolf Minkowski correctly found that the center of the Crab Nebula contained an unusual star.\cite{Baade42, Minkowski42}
In the following years, the mystery gained depth when a radio emission was also detected.\cite{BSS49}
This gathered a lot of interest from the theorists, as they were trying to explain the origin of the energy powering the nebula.
In 1953, Shklovsky was on the right track again when he predicted that the emission is due to synchrotron radiation from relativistic electrons spiraling along magnetic field lines. 
His predictions where further strengthened in the next year 1954, when Victor Dombrovsky discovered that the optical radiation from the Crab nebula is polarized,\cite{Dombrovsky54} as it should be if the radiation originates from synchrotron process.
The next piece of the puzzle came in 1964, when Lodewijk Woltjer, who did his PhD on the Crab Nebula, argued, based on the conservation of magnetic flux, that neutron stars should have a strong magnetic field, enough to produce this synchrotron radiation.\cite{Woltjer64}
Similar results were independently obtained in the East by Vitaly Ginzburg.\cite{Ginzburg64}


Early X-ray telescopes of the time had a very poor angular resolution, so imaging the Crab Nebula to get an answer to the puzzle was difficult.
The first observation in 1964 by S. Bowyer et al. used a clever method of partial lunar occultation to cover unwanted parts of the sky with the Moon, and what followed was the first X-ray observation of the neutron star candidate everybody was waiting for.\cite{BBC64a}
It was, however, followed by a disappointment when a follow-up observation measured the source size to be about 1 light-year in size ($10^{13}\km$) in comparison to the 11 light-years of the whole nebula.\cite{BBC64b}
The result was much larger than what was expected for a neutron star that should be a mere $\osim 10\km$ in radius.
Ironically, what they did not know was that this was just as expected;
for young neutron stars like the one in the Crab Nebula, a pulsar wind (consisting of charged particles similar to solar wind) is expected. 
This wind will then create a surrounding shell called a plerion, much bigger in size, around the neutron star, and this shell is the source of the X-rays.
Hence, the mystery remained even though Nikolai Kardashev in the East and Franco Pacini in the West gave plausible pioneering explanations for the formation of the wind in 1964 and 1967, respectively.\cite{Kardashev64, Pacini67}


Despite all the efforts (and partly due to bad luck), no concrete observations supporting the existence of neutron stars still existed.
This all changed in July 1967, in the farmlands near Cambridge.
There, a pasture was filled with a primitive antenna consisting of wires hanging from stakes --- a state-of-the-art radio antenna of those times.
The idea was to use this newly build radio telescope to study interplanetary scintillation that could help in resolving quasars, another form of compact objects powered by black holes, from extended sources in the sky.
Among several other students who were working for Anthony Hewish was a young PhD student named Jocelyn Bell.
In addition to the signal from the scintillation, she discovered a deviation on her chart-recorded papers; 
an extremely regular signal with a period of 1.3373012 seconds caught Bell's attention.
Originally, this was dubbed (partially as a joke) Little Green Men 1 (LGM-1).
In reality, what they were seeing, Bell quickly realized, was the first pulsar, a rapidly rotating neutron star whose radio emission beam sometimes points towards us, like a distant lighthouse.
More Little Green Men quickly followed, and by the end of the year 1968, dozens of LGMs were known.
The finding was later published in the Nature of 1968 by Hewish et al.\cite{Hewish68}
Hewish's announcement was quickly followed by more than 100 papers on pulsars, speculating the possible origin of the signal.
The winning argument came from Timothy Gold, who showed that pulsars are strongly magnetized rapidly rotating neutron stars.\cite{Gold68}
However, one should not forget the similar seminal theoretical paper already made in 1967, before the discovery, by Pacini.\cite{Pacini67}
More proof came when our old friend the Crab Nebula was shown to host a pulsar rotating at a period of merely $33$ milliseconds.\cite{CCL69}
Anything but a neutron star would be destroyed by the centrifugal forces from such rotation.

The finding of Bell and Hewish was sensational and marked the first detection of a neutron star, almost 40 years after the theoretical speculation by Landau.
Later on, Hewish was awarded the Nobel Prize in Physics in 1974 "for the discovery of pulsars", a somewhat unfair recognition taken into account that it was Bell who found them in practice.
Hence, despite all the efforts in X-ray astronomy, the concluding evidence finally came from the radio wavelengths.


One should not, however, feel sorry for the X-ray astronomers, as they got their fair share of neutron-star-related revelations during the next decade.
Important discoveries especially for studying the nature of accretion, or how matter infalls onto a compact object, came from the first long-duration observations done with the Dutch astronomy satellite ANS.
As a direct competitor for the European ANS, the U.S. funded Los Alamos nuclear research center was also in the game of observing X-rays from compact objects.
Their Vela satellites were sent to space mostly to monitor the compliance of the 1963 Partial Test Ban Treaty of nuclear weapons but they were used for science, too.
In 1975, the ANS satellite was commissioned to study possible black holes in the center of globular clusters but happened to stumble upon something completely different;
Short, $\osim 60$-second-long X-ray flares were detected from the globular cluster NGC 6624 by Grindlay and Heise.\cite{GGS76}
The competing Los Alamos group found similar energetic bursts, but due to the poor angular resolution (collecting X-rays from the Earth was easy and hence no effort was put in for a good spatial accuracy), they could not pin point the exact location of the sources.\cite{BCE76}
Later on, Clark et al. went through the existing SAS-3 data from May 1975 and found a series of ten similar bursts from the same location, NGC 6624.\cite{CJB76}
Even more retrospectively, it turned out that these strange flares had already been observed in 1969 from Cen X-4\cite{BCE72} with another Vela satellite and in 1971 with the Soviet Kosmos 428 X-ray detector\cite{BKM75}.
Their nature, however, remained elusive.

Pioneering theoretical work on thermonuclear instabilities on the surface layers of accreting neutron stars was initiated by Hansen and van Horn in 1975.\cite{HvH75}
They constructed stationary burning shells to lay on top of neutron stars but instead found out that most of them were actually unstable.
The choice of word, unstable, might not convey the full weight of the physical issue though;
such a layer on top of the surface of a neutron star burning uncontrollably meant a spectacular firework.
Shortly after the Los Alamos results came in, an Italian astrophysicist Laura Maraschi was able to connect the dots while visiting MIT in February 1976 and speculated that these recently observed X-ray bursts were due to thermonuclear flashes on the surface of accreting neutron stars.\cite{MC77, Lewin93}
Woosley and Taam concluded similarly in their 1976 paper titled ``Gamma-ray bursts from thermonuclear explosions on neutron stars''.\cite{WT76}
Observational evidence soon followed when van Parajids et al. and Thorstensen et al. were independently able to optically resolve the companions of two known bursting sources, Cen X-4\cite{vPV80} and Aql X-1\cite{TCB78}.
Not only did these observations confirm that there is a companion star close by but also that it must be within such a close distance of the neutron star that accretion, i.e. a constant flow of new fuel for the explosions, can exist.

All of the aforementioned discoveries were, however, nothing but a prelude to what was discovered in the years to follow.
We will end this short historical review by listing some of the most important more modern findings.
A big revelation came in 1979 when a very intense burst of gamma rays was detected by two Soviet satellites, Venera 11 and Venera 12.\cite{MGI79}
Later dubbed Soft Gamma Repeaters (SGRs), their energy source remained mysterious for decades.
A theoretical breakthrough came in 1992 when Robert Duncan and Chistropher Thompson showed that the bursts, orders of magnitude stronger than the X-ray bursts, could originate from a neutron star with a magnetic field $100$ to $1000$ times more powerful than what was previously known.\cite{DT92}
Today, these neutron stars are more commonly known as magnetars, a subclass of young neutron stars where the initial magnetic field has been amplified by delicate dynamo processes during the supernova explosion.
Another surprise came in 1982, when a team led by Donald Backer changed how we look at pulsars when, using the world's largest radiotelescope in Arecibo, they found a pulsar spinning $641$ times per second.\cite{BKH82}
This new neutron star was dubbed a millisecond pulsar, and unlike its predesessors, we now know that instead of slowly decreasing in spin, it belongs to a class of old pulsars that have been spun up by the accretion.
In 2000, our understanding of the thermonuclear X-ray bursts was also changed when Cornelisse observed a very long, not minutes but hours long, burst from a neutron star normally exhibiting regular short bursts.\cite{CHK00}
These were then dubbed as superbursts, in contrast to normal ones.
The reason for this difference is, we think, the burning material;
normal bursts use hydrogen and helium as their fuel but superbursts can devour a carbon shell in a matter of hours if the conditions are just right.



\section{From first principles to a neutron star}

\subsection{Background: Sun and stars}
First, let us see what we can learn from neutron stars using simple estimates and conservation laws.
Neutron stars are born from the death of a normal star.  %\mnote{Stellar orders of magnitude}
The most familiar one to us is our Sun, one Astronomical Unit or $\Ten{1.496}{13} \cm$ away from us.%
\footnote{Throughout this thesis, we will typically present our quantities only up to some fixed precision instead of the full litany of numbers.
We will also adopt the centimeter-gram-second (cgs) unit system instead of the (maybe) more common SI-system. 
Such a selection is sure to disappoint some, but try to endure.
}
With a mass of $\Msun = \Ten{1.99}{33} \g$ and a radius of $R_{\odot} = \Ten{6.96}{10} \cm$, our Sun gives us an idea of the typical stellar scale.
Curiously, these numbers also mean that the mean density of the Sun is $\rho_{\odot} \approx 1.41 \gcm$, only $1.4\times$ the density of water.

Like all stars, our Sun is held together by the inward-pulling gravity.
Gravity does not prefer any direction more than another, and so a spherical object is expected to form.
In addition to the inward-facing force, an outward-facing force is needed to balance the system.
For normal stars this force originates from thermal gas pressure.

We observe stars in the night sky because they shine.
This radiation originates from the thermonuclear fusion reactions inside the star.
\emph{Thermo} here refers to the temperature and heat, \emph{nuclear} to the atomic nuclei, and \emph{fusion} to a process where elements are fused together.
During the thermonuclear fusion process, the star's core fuses light elements such as hydrogen into heavier ones like helium.
The mass of four hydrogen atoms is more than a mass of one helium atom.
This mass difference between the start and the end results is then transferred into energy in accordance to Einstein's famous $E = mc^2$ formula.
A whole sequence of such fusion processes takes place inside the star, where lighter elements are merged together to build heavier and heavier elements.
The energy release from this mass-to-energy conversion will then give the star a sufficient thermal pressure support to keep it from collapsing under the relentless gravity trying to squash it.

The fusion of elements does not continue forever.
In the beginning, four protons collide to form an alpha particle%
\footnote{Alpha particle consists of two protons and two neutrons, i.e., doubly ionized helium nuclei.}
In the next stage, three helium nuclei collide to form carbon, and so on, until iron is created.
Production of iron marks the end of the possible fusion chain, since the fusion of two iron atoms no longer releases energy.
On the contrary, it requires external energy to take place.%
\footnote{This opens up another possibility of creating energy by splitting heavy elements, an inverse process to what is described here. Such a process is called fission and is familiarly taken advantage of in Earth's nuclear power plants.}
This iron produced will then form a dead core without any energy output.

Like all big furnaces, at some point the star will run out of fuel to burn.
What is left behind is an inner core of iron with subsequent onion-like layers of lighter and lighter elements.
The crucial question to ask next is: 
what is supporting this iron core now that the thermal gas pressure from the fusion process is lost?
This was the question that led scientists like Chandrasekhar to the realization of degenerate matter and white dwarf stars in the 1920s.

%To answer this, we need to look inside the iron atom.
%An iron nucleus, consisting of $26$ protons and $32$ neutrons and surrounded by $26$ electrons, repulses its neighbors because of the negatively charged electron cloud around it does not want to get in touch with its neighbors in the iron-atom lattice.
%\footnote{A more precise consideration shows that nickel ($^{56}_{28}\mathrm{Ni}$) is actually thermodynamically more favored in the core because of the lack of neutrons needed to synthesize iron-58 ($^{58}_{26}\mathrm{Fe}$).
%The underlying idea presented here, however, remains the same.
%}
%It is this antisocial avoidance of neighboring particles, originating from electric charge repulsion, that will then give the internal support for the iron core to not collapse under its own gravitational pull.
%If enough iron is built up during the lifetime of the star so that even this repulsion force is not enough, we can continue our thought experiment and ask again: what will follow?
%This was the question that led scientists like Chandrasekhar to the realization of degenerate matter and white dwarf stars in the 1920s.

\subsection{White dwarfs and quantum mechanics}
The answer lies in the elusive quantum mechanics.
When the atoms inside matter are packed close enough together, we need to apply wave-like characteristics for them instead of classical point-like thinking.
Because of their smaller mass, the electrons orbiting the nuclei enter the realm of quantum mechanics first, in comparison to the heavier protons and neutrons in the atomic core.
A freely moving electron confined into a small enough space because of its surrounding neighbors will start to attain only some fixed values of momenta.
In physics, we speak about the quantization of energy levels.
The reason for this is similar to a vibrating string of a guitar; a string fixed from both ends can only vibrate on some specific wave modes that are set by its length.
An additional complication for the electrons is set by the Pauli exclusion principle, which forbids more than one electron to occupy the same wave mode or quantum state inside the same region.
%\footnote{\red{Explain Pauli repulsion for the layman}}
This gives rise to a degeneracy pressure as electrons fill their quantum states from the lowest to the highest, and can thus not be packed any tighter together.
A star held together by this degeneracy pressure of its electrons is known as a white dwarf.
From this setup, it only takes a short step into realizing the existence of neutron stars, because we can, once again, push forward and ask: what next?%
\footnote{The reader can be assured that the chain of thermal pressure $\rightarrow$ charge repulsion $\rightarrow$ electron degeneracy will come to a halt as the final neutron degeneracy really is the last possible supporting force in nature (maybe excluding quark matter, though\ldots)}


\subsection{Neutron stars, at last}
What if, at some point, even these quantum effects of the electrons are not enough to support the star?
One does not need to worry, since after the lightweight electrons have given all they can, it is the heavy neutrons that slowly start to enter the quantum mechanical realm.
In practice, the matter will turn into a one big team of neutrons because when the positive ($+$) central proton and the surrounding negative ($-$) electron come in contact, a neutral neutron is created.%
\footnote{In reality, the beta decay formula is $e^- + p = n + \bar{\nu_e}$, where the additional electron anti-neutrino is needed to preserve the quark color neutrality.}
The degeneracy pressure of such a neutron porridge is multiple orders of magnitude larger than what the electrons can offer, yielding an ultimate solution to the pressure support problem.

Let us consider the consequences of this thought experiment.
More detailed calculations show that the resulting iron core sitting at the center of the star is weighing a maximum of around $\osim \Msun$.%
\footnote{This is quite a reasonable-sounding assumption considering that the stars that explode are around $\osim10\Msun$ in size and we certainly do not expect everything to fall into the core.
}
Hence, there are $M_{\mathrm{core}} / m_{\mathrm{atom}} \sim \Msun / \mproton \approx \Ten{1.99}{33} \g / \Ten{1.67}{-24} \g \sim 10^{57}$ atoms trapped inside the core.
\footnote{
    The mass of the atom, $m_{\mathrm{atom}} = \mproton + \me$, is approximated (to an excellent accuracy) by only considering the central nuclei alone as the electron mass $\me \approx \Ten{9.11}{-28} \g$ is negligible in comparison to the proton mass.
}
Here we are already considering not iron atoms but pure hydrogen atoms only to simplify the presentation.
Working backwards from these numbers, we can estimate the size of the compressed core.
Using a typical radius of $r_{\mathrm{n}} \approx \Ten{1.25}{-13}\cm$ for the nuclei, we would expect these particles to form an object of around $R \sim (10^{57})^{\frac{1}{3}} \times \Ten{1.25}{-13}\cm \sim 10^6 \cm$. Thus, we have ended up with a star consisting of only neutrons, with a size of $\osim 10\km$ and a mass of $\osim 1\Msun$: a neutron star!

By considering simple order-of-magnitude estimates we have now ended up characterizing the dimensions of a typical neutron star. %\mnote{Density}
A canonical neutron star is often taken to have $R=10\km$ and $M=1.4\Msun$, so let us also adopt these numbers for the following considerations.
Such dimensions give us an impressive mean density of $\rho \sim \Ten{7}{14} \gcm$.
In comparison, for a typical nucleon (such as a neutron or a proton) we had $m_{p} \approx m_{n} \approx \Ten{1.67}{-24} \g$ and $r_{n} \approx \Ten{1.25}{-13} \cm$, yielding us a nuclear density of $\nsat \approx \Ten{2}{14} \gcm$.
Not surprisingly, the densities are of similar magnitude.
However, when comparing these numbers to our everyday matter, the difference is huge, almost $14$ orders of magnitude;
a cubic centimeter of water weighs $1\g$, whereas the same volume of neutron star matter would weigh $100\,000\,000\,000\,000\g$ or $100$ million metric tons.


Matter compressed to such a small volume has an extreme impact even on the surrounding spacetime. %\mnote{Escape velocity}
Let us try to estimate, again, the order of magnitude of these effects by considering the escape velocity --- a velocity needed to escape the local gravitational pull of an object.
For us, on the surface of the Earth, it turns out to be $v_{\mathrm{\Earth}} = \sqrt{2G M_{\Earth} / R_{\Earth}} = \Ten{1.12}{6} \cms$, for $M_{\Earth} = \Ten{5.97}{27} \g$ and $R_{\Earth} = \Ten{6.37}{8} \cm$.
Similarly, for the Sun it is $v_{\odot} = \Ten{6.18}{7} \cms$, or $0.002 \times$ the speed of light.
On the other hand, for a neutron star, we obtain $v_{\mathrm{NS}} = \Ten{1.93}{10} \cms$, which is already about half of the speed of light!
Hence, relativistic effects become crucial to take into account when considering neutron stars, as one can not even escape from the surface of the star without velocities close to those of the light.


Let us next think about the possible spin rates that a neutron star can have. %\mnote{Spin}
For our Sun, it takes about one month (or approximately $25.5$ days, to be more exact) to revolve around itself, corresponding to a spin rate of $\Ten{4.5}{-7}\Hz$.
When compressed to the dimensions of a neutron star, the radius changes by a factor of $R_{\odot} / R_{\mathrm{NS}} \approx \Ten{6.96}{10}\cm / 10^6 \cm \sim \Ten{7}{4}$.
It is important to notice that when a rotating object collapses, it preserves it's angular momentum, not the spin rate.
Similar to an ice-figure skater pulling her arms inwards while spinning, we observe an increase in the spin in order to preserve the angular momentum.
As the rotational inertia increases as a square from the distance to the axis, our Sun, when compressed to the scale of a neutron star, would obtain a spin of $\Ten{4.5}{-7}\Hz \times (\Ten{7}{4})^2 \sim \Ten{2}{3} \Hz$, $2000$ revolutions per second.
The young proto neutron star, however, quickly slows down after its birth, so more typically, spins of around $100$ to $1000 \Hz$ are observed, which is still about one revolution per $1$ to $10$ milliseconds.


One final characteristic we can try and estimate is the magnetic field. %\mnote{$B$-field}
Here we can follow a similar chain of reasoning as with the spin and start from typical values such as those of our Sun.
For the Sun, the slow rotation gives rise to a dynamo process that produces a magnetic field of around $B_{\odot} \approx 1\Gauss$.%
\footnote{A typical refrigerator magnet is about $50\times$ stronger with a magnetic field of $50\Gauss$.}
When considering magnetic field, it is the magnetic flux through the surface that conserves, hence we expect the field to scale also as a square of the radius.
Using the same compression ratio of \Ten{7}{4} for the radius, we then obtain $B_{\mathrm{NS}} \approx 1 \times (\Ten{7}{4})^2 \Gauss \sim 10^{10} \Gauss$.
Comparing this to the value of $10^6\Gauss$ for the strongest non-destructive magnet on Earth, we start to grasp the level of energetics that neutron stars have to offer; even their original non-amplified magnetic field is $\times 10\,000$ stronger.
In some cases, a dynamo effect originating from the rapid rotation of the star can amplify the magnetic field even by a factor of a million. 
This gives rise to neutron stars with immense magnetic field strengths of $B \sim 10^{16} \Gauss$.

It is fair to conclude that neutron stars are dominating the record tables of physics in almost all of their aspects.
They are \emph{super}dense, \emph{super}fast rotators, sources of \emph{super}strong magnetic fields, and \emph{super}rich in the range of physics involved.
In short: they are the \emph{super}stars of physics!%
\footnote{
    This is (humorously) called the Pines theorem as everything is \emph{super}- when considering neutron stars, as postulated by David Pines in a talk given at the conference on Neutron Stars: Theory and Observation (The NATO Advanced Study Institute, Crete, Greece, September 3–14, 1990).
}
