\chapter{Introduction} 
\chapterimage[width=15cm]{wordcloud/chap1b.png}
Neutrons stars are curious objects encompassing many still unsolved problems of modern physics and astrophysics.
Their unique nature makes them ideal laboratories for many of the most energetic phenomena in space.
Born from the ashes of a supernova, they begin their life only when some other normal star fades away and dies in a spectacular supernova explosion.%
\footnote{ For the pedantic ones, we specify that a supernova is actually an implosion followed by a subsequent explosion.}
From there on, they continue their life by slowly collecting the surrounding interstellar matter or by devouring an unlucky companion star floating next to them.
It is not the impressive $\sim 10^{30}$ kilograms they weight but the mere $\sim 10 \km$ in radius sphere that they encapsulate this material into that is then able to bend the spacetime itself.
Such an impressive feat rewards them with a categorization into a special stellar group called \textit{compact objects}.
Along with white dwarfs and black holes, these strange compact objects have been under a scientific scrutiny for almost nine decades now.
Still, some of the most fundamental questions remain:
What are neutron stars made of?
How big are they?
How do we see them?

In the next few sections, we will discuss these peculiar objects habiting the Space around us in detail.
First, a short history of their discovery is given, followed by simple physical arguments why we actually expect them to exists in a first place.
This will hopefully help us in building some intuition of physics that are possible near and inside these stars.
Next, some actual observable phenomena is reviewed.
These are also closely connected to the surroundings and environments of the neutron stars, and hence those will be discussed and described also.


\section{Short history}
\subsection{From imagination into a reality}

%Neutron stars
%Anticipation by theorist before the second World War. \cite{Landau32}
%Neutron anticipated by Rutherford.\cite{Rutherford20}
%Possible existence\cite{Chadwick32a}
%Existence\cite{Chadwick32b}
%Nobel prize 1935

In 1908, Lev Landau was born in Baku, Azerbaijan.
Lev turned out to become a brilliant Soviet scientist who already at an age of 14 matriculated from the Baku State University studying in both, Department of Physics and Mathematics, and the Department of Chemistry simultaneously.
Soon after that he became disinterested in chemistry but continued to be fascinated by physics through his whole career.
In 1931, at an age of 23, Landau published an exceptional paper where he speculated on the existence of stars containing matter on nuclear densities:\cite{Landau32}
%In 1931, at an age of 23 years, a Soviet physicist Lev Landau published an exceptional paper where he speculated on the existence of stars containing matter on nuclear densities:\cite{Landau32}
\begin{displayquote}
    \textit{...the density of matter becomes so great that atomic nuclei come in close contact, forming one gigantic nucleus.}
\end{displayquote}
What made this paper exceptional was that it was done before the existence of neutrons were confirmed, hence proposing that quantum mechanics might be violated as the atoms consisting of protons and electrons only would live together.
Nevertheless, it marked the first theoretical speculation on the existence of what we now know as neutron stars.

Lev needed not to wait for long.
It was already next year in 1932, that James Chadwick then confirmed that neutrons really were a fundamental part of our nuclear physics with his works dubbed as \textit{Possible Existence of a Neutron}\cite{Chadwick32a} and a later follow-up of \textit{Existence of a Neutron}\cite{Chadwick32b}.
His findings then became to confirm the theoretical predictions of his supervisor Ernest Rutherford made already in 1920\cite{Rutherford20}.
Later on Chadwick was awarded the Nobel Prize in physics (1935) of his findings.
Chadwick himself continued his career as part of the Manhattan project, as it was basically his work that inspired the U.S. government to begin a serious atomic bomb research.

%Speculated in December 1933 by Baade and Zwicky in Meeting of the American Physical Society at Stanford.\cite{Baade34a}
%and later published in Phys. Rev. \cite{Baade34aa}
%\begin{displayquote}[Baade och Zwicky]
%With all reserve we advance the view that a super-nova represents the transition of an ordinary star into a neutron star, consisting mainly of neutrons. Such a star may possess a very small radius and an extremely high density.
%\end{displayquote}
%Was deemed work of imagination from weird theorists.
%A.G.W. Cameron, recalling his post-doc academic year 1959-1960 at Caltech reminds (Cameron, 1999): 

Now that the existence of the neutron was confirmed, it did not take long for others, independent of Landau, to propose similar stars.
One year after the discovery, in December 1933 at the Meeting of the American Physical Society at Standford, Wilhelm Baade and Frank Zwicky made their famous propose that Supernovae should be considered as a new category of astronomical objects.\cite{Baade34a, Baade34aa}
At the same time they also stated:
\begin{displayquote}
    \textit{...we advance the view that a super-nova represents the transition of an ordinary star into a neutron star, consisting mainly of neutrons. Such a star may possess a very small radius and an extremely high density.}
\end{displayquote}
Such a statements were, however, deemed as a work of imagination from a bunch of weird theorists.
Zwicky, on the other hand, kept on insisting that neutron stars are really out there.
A.G.W. Cameron, a former post-doc during the 1959$-$1969 at Caltech where Zwicky was also situated, recalls:
\begin{displayquote}[Cameron, 1999]
    \textit{For years Fritz [Zwicky] had been pushing his ideas about neutron stars to anyone who would listen and had been universally ignored. 
    I believe that the part of the problem was his personality, which implied strongly that people were idiots if they did not believe in neutron stars.
    }
\end{displayquote}


%\subsection{History of degenerate electron gas and Chandrasekhar limit}
%electrons as fermions\cite{Dirac25}
%electron deg. pressure \cite{Fowler26}
%elecs must be relativistic \cite{Anderson29}
%equation of state for uniform density \cite{Stoner30}
%Similar independent study by Landau \cite{Landau32}
%full equation of state of degenerate electron gas unaware of Fowler \cite{Frenkel28} \cite{Yakovlev94}
%Finally put together by Chandrasekhar \cite{Cha31} work done 5 years before Stoner.
%
%Baade and Zwicky were apparently unaware of the work about the maximum mass of white dwarfs. 
%In 1930 Subrahmanyan Chandrasekhar applied Einstein’s theory of Special Relativity to the stellar structure when he was only 20, and developed the theory of white dwarfs (he was awarded the Nobel prize in 1983).
%With John von Neumann, they obtained in 1934 the equations describing static spherical stars in Einstein’s theory of General Relativity but they didn’t publish their work.\cite{Baym82}

Progress on the theoretical understanding of neutron stars was also tightly connected to understanding the interiors of white dwarfs.
Unlike the mysterious nuclear forces related to neutrons, white dwarfs physics was more related to understanding the behavior of electrons.
A breakthrough came in 1925 when a young Paul Dirac formulated the quantum wave equations for the motion of the electrons\cite{Dirac25}.
What soon followed, was a description of a pressure of degenerate electron gas by Ralph Fowler, Dirac's supervisor\cite{Fowler26}.
Implications of this were severely against the previously known physics:
Even in zero temperature, there would be a degeneracy pressure preventing matter from collapsing caused by the exclusion principle of quantum mechanics.

Using a simplified uniform density approximation, Edmund Stoner was then able to show that this implied a maximum mass limit for white dwarfs.\cite{Stoner30}
Thus, a surprising result was obtained: when the density of a white dwarf approaches infinity, the mass reaches a maximum value.
It was later on realized by German-Estonian Wilhelm Anderson that the electrons in this problem must actually be treated relativistically\cite{Anderson29}, something omitted by Stoner.
Anderson also tried to correct the crude mistake by deriving the equation of state of relativistic degenerate electron gas but ended up making severe mistakes.
It was then Stoner who corrected his equations, based on the communications with Anderson, and re-derived his maximum mass limit.
Regardless of Stoner's efforts, the maximum mass limit was later named Chandrasekhar's mass limit for its importance in astrophysics.

This was to honor Subrahmanyan Chandrasekhar, a young prolific Indian physicist and astrophysicist who was working on the same topic after reading Fowler's paper on the degenerate electron gas.
Unlike Stoner's limit computed using the uniform density approximation, Chandrasekhar realized that a polytropic density profile is a more physical, albeit mathematically more challenging, formulation.
19 years old Chandrasekhar, already known for his mathematical skills, was still able to numerically integrate the equations by hand and obtained a similar limiting mass\cite{Cha31}.
Later on, it has been, however, found that Chandrasekhar was not even the second person to derive the mass limit, but third:\cite{Yakovlev94}
A soviet physicist Yakov Frenkel published a similar derivation, independently and unknowingly of the progress in the west, where he applied the relativistic degenerate electron gas results to white dwarfs, and concluded that a similar upper limit on the mass exists\cite{Frenkel28}.
It was, however, due to the slow publishing speed in the Soviet journals, that his work never became available to his western peers.

%Connection between white drwards and neutorn stars.
%It was Gamow who first made the connection in 1939. \cite{Gamow39}
%At a conference in Paris in 1939, Chandrasekhar also pointed out
%\begin{displayquote}
%If the degenerate core attain sufficiently high densities, the protons and electrons will combine to form neutrons. 
%This would cause a sudden diminution of pressure resulting in the collapse of the star to a neutron core.
%\end{displayquote}
%A neutron star should thus have a mass close to the Chandrasekhar limit, i.e. $M \sim 1.4\Msun$.


Nevertheless, a maximum mass for a white dwarf was laid out, and in the end after all the relevant physical inclusions, it turned out to be $1.44~\Msun$, or $1.44$ times the mass of our Sun.
What makes this limit even more important is that a maximum mass for a white dwarf is just an minimum mass for a neutron star.
An important connection first made by George Gamow in 1939\cite{Gamow39}.
Idea behind this is simple: 
If it is the degenerate electron gas pressure, quantum mechanical in nature, that keeps the white dwarfs from collapsing, what happens when the maximum mass is attained and even this pressure is note able to resist the forces of gravity?
At a conference in Paris in 1939, Chandrasekhar then laid out the answer:
\begin{displayquote}
    \textit{
    If the degenerate core attain sufficiently high densities, the protons and electrons will combine to form neutrons. 
    This would cause a sudden diminution of pressure resulting in the collapse of the star to a neutron core.
    }
\end{displayquote}
A neutron star should thus have a mass close to the Chandrasekhar limit, i.e. $M \sim 1.44\Msun$, and consists of neutron only.
Exactly like proposed by Landau eight years ago without knowing anything about neutrons yet, or later on by Baade and Zwicky when they presented their theory of supernovae!


%TOV
%Global structure was reveled in 1939.
%Richard Tolman \cite{Tolman39}
%and Robert Oppenheimer \& his student George Volkoff \cite{OV39}
%reobtained independently the equations describing static spherical stars in General Relativity.
%Oppenheimer and Volkoff solved these equations and calculated numerically the structure of non-rotating neutron stars.
%Oppenheimer and Volkoff found $M_{\mathrm{max}} \approx 0.7\Msun$ by considering a degenerate gas of free neutrons.
%Since this is smaller than the maximum mass of supernova cores, they erroneously concluded that neutron stars could not exist.


It was then before the second World War, that a solid basis for a theory of neutron stars was laid out.
This was, however, just the beginning. 
Next question would be the critical one that we are still trying to find the answer to:
If they exists, how big are they?
The problem was that because of the extremely dense nature of these objects, the standard stellar equilibrium equations were not valid anymore, and because of this, it was not possible to estimate even the size of the star.
The problem was unwieldy because it was general relativistic in nature:
The immense mass of the neutron star was bending the spacetime itself, and more compact it was, the more it could bend it. 
On the other hand, the more curved the spacetime was, the more the star would gain weight.

It was already during the same year in 1939 as Gamow's remark, that a theoretical framework for studying this was published.
This was done independently by Richard Tolman\cite{Tolman39}, and Robert Oppenheimer together with his student George Volkoff\cite{OV39}.
Both papers were even submitted on the same day, 3rd of January, to Physical Review and were published on the same February issue.
More importantly, they both described a hydrostatic equilibrium for a spherically symmetric object in general relativity, exactly what is needed to study neutron stars.
Because of the utter importance, the solution is now known as Tolman-Oppenheimer-Volkoff equation.
In addition, Oppenheimer and Volkoff applied their equation and numerically calculated a structure of a neutron star consisting of non-interacting strongly degenerate neutron gas.
This marked the first try in characterizing neutron stars.
Similar to white dwarfs, they also obtained an upper limit for the mass. 
As a disappointment for everybody, it was, however, calculated to be of around $0.7\Msun$, i.e. less than the Chandrasekhar limit of $1.44\Msun$ for white dwarfs, indicating that neutron stars could not exist in nature.
It took almost two decades then to show that it was actually the assumption of no interaction between the neutrons that was causing this hiccup.

More over, it was not actually Tolman, or Oppenheimer and Volkoff who first discovered general relativistic hydrostatic equation.
It was now Chandrasekhar's turn to avoid having an important result credited to him:
Together with John Von Neumann, Chandrasekhar extended his work on white dwarfs to cover also neutron stars and in the process derived exactly the same equilibrium equation in 1934, i.e., five years before the groundbreaking publication of Tolman, Oppenheimer and Volkoff.\cite{Baym82}
It is, however, worth mentioning that later on, in 1983, Chandrasekhar received a Noble prize in physics for his work on ``theoretical studies of the physical processes of importance to the structure and evolution of the stars''.
So he certainly received at least some credit from his important work.

% about energy production
%In 1937, Gamow and Landau proposed independently that a possible stellar energy source could be the accretion of matter onto a dense neutron core.
%\red{source, image from K.S. Thorne?}
%But very soon it was shown that stars are powered by thermonuclear reactions (as suggested in the 20s by Eddington and others). 
%The interest in neutron stars then faded away.
Around the same time, in 1937, Gamow and Landau also independently proposed that a accretion of matter onto a dense neutron star core could be the missing source of energy for stars.
This increased the interest towards neutron stars and the field flourished on the 1930s.
Soon it was, however, shown that stars are powered not by accretion but by thermonuclear reactions (as suggested in the 1920s by Eddington and others).
The interest in neutron stars then faded away and the research focused on weaponing the nuclear forces.


%Eos
%The first “realistic” EoS of dense matter was constructed in the 50s by John Wheeler and his collaborators.\cite{Wheeler66}
%For the crust, a semi-empirical mass formula was used together with the EoS of degenerate electrons. 
%In the core, matter was assumed to be a mixture of three ideal Fermi gases (neutron, proton and electrons).
%In 1959, Cameron constructed neutron-star models using the Skyrme equation of state for high-density matter. 
%nuclear forces considerably stiffen the EoS
%He found that $M_{\mathrm{max}} \approx 2\Msun$.
%neutron stars can thus be formed as proposed by Baade and Zwicky
%neutron star cores may contain various nuclear species such as hyperons.

Next big breakthrough came almost 20 years later in the 1950s, when John Wheeler and his collaborators constructed the first realistic equation of state of dense matter\cite{Wheeler66}.
For the outer layers, known as the crust, they applied a semi-empirical mass formula together with the equation of state of degenerate electrons.
For the dense core, they assumed a mixture of three ideal Fermi gases compose of neutrons, protons, and electrons.
This marked the first consistent formulation for the neutron star structure.
It was followed by Cameron who applied the Skyrme equation of state for the high-density matter.
This had important conclusions, as he was then able to show that nuclear forces stiffen the matter considerably in comparison to the non-interacting neutrons.
Similar to Tolman and Volkoff, he then went on and calculated the maximum possible mass of approximately $2\Msun$.
This marked an important theoretical breakthrough as it implied that neutron stars can, after all, exists.
A new wave of interest towards neutron stars was thus launched as everybody wanted to observe them.


\subsection{Many observational faces of neutron stars}
%Hope of observing them
%Formed in supernova explosions, neutron stars were thus expected to be "hot". 
%In the 60s, theoretical efforts focused on modeling the cooling of neutron stars motivated by the hope of detecting their thermal emission.
%First cooling calculations predicted surface temperatures $T \sim 10^6$ K for neutron stars $\sim 103$ year old.\cite{CS64}
%a neutron star emits mainly in X-rays. 
%So neutron stars were not expected to be seen from Earth because X-rays cannot penetrate the atmosphere.

After Wheeler and Cameron laying the modern foundation for the neutron star structure studies, everybody was eager to find these strange objects from the night sky.
It did not take long, when researchers realized that as neutron stars are born in the supernova explosions, we expect them to be hot.
Most of the theoretical effort in the 60s was then focused on developing models for the cooling of neutron stars\cite{Stabler60, Chiu64, Morton64, CS64, BW65a, BW65b, TC66}
It was the possible thermal radiation from this cooling that could then be used to detect them, as was first shown by Hong-Yee Chiu\cite{Chiu64}.
First calculations predicted surface temperatures of $T \sim 10^6 \Kelvin$ for a neutron star of around 1000 years old.
This had important implications for the observers as it means that neutron stars would radiate mainly in the X-rays.
Atmosphere of Earth, on the other hand, was impenetrable in the X-ray wavelengths.
Luckily, 60s also marked the beginning of golden era for spaceborn observatories.

%Observations
%X-ray observations in space started in the 60’s with pioneer experiments by Riccardo Giacconi (Nobel Prize 2002).\cite{GGP62}
When X-rays did not reach the earth, humankind went to space.
In the late 1950s and early 1960s it was the pioneering experiments of italian astrophysicist Riccardo Giacconi that opened this new window the Universe.
Giacconi started first with rocket-borne experiments and later on continued by leading the development of the first orbiting X-ray satellite Uhuru, ``freedom'' in Swahili.\cite{GGP62}
After the first X-ray satellite, Giacconi then continued with the Einstein Observatory, the first fully imaging X-ray satellite, and later with Chandra X-ray observatory.
For all of his efforts, he received the Noble Prize in Physics in 2002 ``for pioneering contributions to astrophysics, which have led to the discovery of cosmic X-ray sources''.

%Several X-ray sources were discovered but their nature remained elusive.
%The activity was also focused on supernova remnant and a natural target was the Crab nebula.
%During the 1920s and 1930s, the Crab nebula was identified as the remnant of a supernova that exploded on July 4, 1054.
%A bright star was observed by Chinese, Japonese and Arab astronomers. 
%The "star" remained visible in daytime for 23 days and disappeared from the night sky after two years.
%Native Americans (Anasazi) might have also observed this event as suggested by the interpretation of a petroglyph in Chaco Canyon.

%In 1967, Iosif Shklovsky correctly proposed that Scorpius X-1 (found in 1962) is a neutron star accreting matter from a normal star. \cite{Shklovsky67}
%But his work attracted little attention among astrophysicists

During the starting boom, several extra-terrestrial X-ray sources were discovered.
Like is common in science, first discovery actually came by accident.
A team led by Giacconi launched an Aerobee 150 rocket in June 1962 to the skies with a payload of highly sensitive soft X-ray detector meant to observe the X-rays from the moon.
Due to a slight change (or a mistake) in the planned trajectory it ended up observing towards the constellation of Scorpius and caught a glimpse of what is now known as an X-ray source Sco X$-$1.
Little did they know that this was actually the first neutron star radiating to us.
Five years later, in 1967 Iosif Shklovsky was first to propose that Scorpius X-1 is a neutron star\cite{Shklovsky67}, but his work attracted little to no attention.

First intended observations to find neutron stars were aimed towards the Crab nebula, a well-known candidate to hosting a neutron star.
The Crab Nebula, already known from 1920s and 1930s to be a supernova remnant, is known to be exploded exactly in 4th of July, 1054.\cite{Oort, Lundmark21, Mayall39, HistSupernovas}
In the contemporary Chinese, Japanese, and Arab history writings, a ``guest star'' is described to appear in the constellation of Taurus, and to persist even in the day-sky for 23 consecutive days.
Even after that, it remained visible in the night sky for two years.
For astronomers, this was a clear sign of a nearby supernova going off.

%A radio emission was detected in 1949.\red{cite} 
%Already in 1942, Baade and Minkowski found that the central region of the Crab nebula contains an unusual star.
%Subsequent theoretical efforts were focused on understanding the origin of the energy powering the Crab nebula.
%In 1953, Shklovsky predicted that this is due to synchrotron radiation by relativistic electrons spiraling along a strong magnetic field. 
%Later the polarition of radio emission was confirmed. 
%In 1964, Lodewijk Woltjer (who did his PhD with Jan Oort on the Crab nebula) argued that neutron stars could have very strong magnetic fields. \cite{Woltjer64}
%This was also independently shown by Ginzurg. \cite{Ginzburg64}

But it was not only the spectacular supernova, but what was left behind that eluded astronomers.
Already in 1942, our old friend Baade and Rudolf Minkowski correctly found that the center of the Crab nebula contained an unusual star.\cite{Baade42, Minkowski42}
In the following years, the mystery deepened when radio emission was also detected.\cite{BSS49}
This gathered lot of interest from the theorists as they were trying to explain the origin of the energy powering the nebula.
In 1953, Shklovsky was on the right track again, when he predicted that the emission is due to synchrotron radiation from relativistic electrons spiraling along magnetic field lines.
Next piece of the puzzle came in 1964, when Lodewijk Woltjer, who did his PhD on the Crab nebula and was supervised by Jan Oort, argued based on conservation of magnetic flux, that neutron stars should have a strong magnetic field.\cite{Woltjer64}
Similar result was independently obtained in the East by Vitaly Ginzburg.\cite{Ginzburg64}

%The Crab nebula was observed during a lunar occultation on 7 July 1964.\cite{BBC64}
%The size of the X-ray source was estimated as 1 light-year $10^{13}$ km (size of the nebula 11 ly). 
%This was much larger than the typical size of a neutron star (10-20 km).
%In 1967, Franco Pacini (who was a young postdoc at Cornell) showed that a rapidly rotating neutron star with a strong dipole magnetic field could power the Crab nebula.
%Nothing concrete, however, came from this for a long time.

Early X-ray telescopes of the time, had a very poor angular resolution so imaging the Crab nebula only to get an answer to the puzzle was hard.
The first observation in 1964 by S. Bowyer et al. used a clever method of partial lunar occultation to cover unwanted part of the sky with the Moon, and what followed was the first X-ray observation of the neutron star candidate everybody was waiting for.\cite{BBC64a}
It was, however, followed by a disappointment when a follow-up observation measured the source size as about 1 light-year in size ($10^{13}\km$) in comparison to the 11 light-years for the whole nebula.\cite{BBC64b}
The result was much larger than what was expected for a neutron star that should be of mere $\sim 10\km$ in radius.
Ironically, what they did not know, was that this just as expected:
For young neutron stars like the one in the Crab nebula, a pulsar wind (consisting of charged particles similar to Solar wind) is expected. 
This wind will then create a surrounding shell, much bigger in size, around the neutron star called plerion, that is the source of the X-rays.
Hence, the mystery remained even though Nikolai Kardashev in the East and Franco Pacini in the West, gave plausible pioneering explanations for the formation of the wind in 1964 and 1967, respectively.\cite{Kardashev64, Pacini67}



%PULSARs
%On July 1967 everything changed for good because of Jocelyn Bell, a graduate student of Anthony Hewish in Mullard Radio Astronomy Observatory.
%Primitive antenna of wires strung on stakes in a pasture.
%
%She discovered that the signal was pulsing with great regularity, at a rate of about one pulse per second. (1.3s)
%Temporarily (jokingly) dubbed "Little Green Man 1" (LGM-1) the source (now known as PSR B1919+21) was identified after several years as a rapidly rotating neutron star.
%Dupped as pulsar, short of pulsating radio source.
%Later published in Nature 1968 by Hewish. \cite{Hewish68}
%The other possibility was that pulsars are strongly magnetised rotating neutrons stars as proposed by Timothy Gold (and earlier by Pacini).
%\cite{Gold68}
%The discovery of the Crab and Vela pulsars definitevely established the nature of pulsars and confirmed the predictions of Baade and Zwicky 35 years earlier that neutron stars are the compact remnants of supernova explosions.

Despite all the efforts (and partly due to bad luck) no concrete observations supporting the existence of neutron stars still existed.
This all changed in the July 1967, in the farmlands near Cambridge.
There, a pasture was filled with primitive antenna consisting of wires that were hanging from stakes.
The idea was to use this newly build radio telescope to study interplanetary scintillation that could help in resolving quasars, another form of compact objects powered by black holes from extended sources in the sky.
Among several other students who were working for Anthony Hewish, was a young post-doc Jocelyn Bell.
In addition to the signal from the scintillation, she discovered a deviation on her chart-recorded papers: 
an extremely regular signal of 1.3373012 seconds caught Bell's attention.
Originally this was dubbed (partially as a joke) as Little Green Men 1 (LGM-1).
In reality, what they were seeing, and what Bell quickly realized, was the first pulsar observed, a rapidly rotating neutron star whose beam sometimes points towards us like a distant lighthouse.
More Little Green Man quickly followed and by the end of 1968 a dozens of LGMs were known.
The finding was later published in the Nature 1968 by Hewish.\cite{Hewish68}

Hewish's announcement was quickly followed by a more than 100 papers on pulsars, speculating the possible origin of the signal.
Winning argument came from Timothy Gold where he showed that pulsars are strongly magnetized rapidly rotating neutron stars.\cite{Gold68}
However, one should not forget the similar seminal theoretical paper already in 1967, before the discovery, from Pacini.\cite{Pacini67}
More proof came when our old friend Crab nebula was shown to host a pulsar rotating at a period of merely $33$ milliseconds.\cite{CCL69}
Anything but a neutron star would be destroyed by the centrifugal forces from such a rotation.

The finding of Bell and Hewish was sensational and marked the first detection of a neutron stars almost 40 years after the theoretical speculation by Landau.
Later on, Hewish was awarded the Noble prize in Physics in 1974 "for the discovery of pulsars", a somewhat unfair recognition taken into account that it was Bell who found them in practice.
Hence, despite all the efforts in X-ray astronomy, the concluding evidence finally came from the radio wavelengths.


%\red{X-ray bursts here}
%Real kick off for X-ray bursts reserach when first long-duration observation were performed with dutch astronomy satellite ANS
%Hunting for a black hole in the globular cluster NGC 6624 Grindlay and Heise stumbled upon X-ray burst 1975 \cite{GH75} \cite{GGS76}
%Competing group in Los Alamos (Belian et al) 1976\cite{BCE76}
%Vela(-5) to monitor compliance with the 1963 Partial Test Ban Treaty by the Soviet Union. Could not make an association to a known X-ray source since their positional accuracy was insufficient.
%Subsequently Clark et al found using exisiting SAS-3 data from May 1975 a series of ten bursts from NGC6624.1820-303 \cite{CJB76}
%Even more retrospectively
%1969 \cite{BCE72} 5 second exponential-like decay Cen X-4, Vela satellite  and later 1971 \cite{BKM75} 3U 1735-44 Kosmos 428  

One should not, however, feel sorry for the X-ray astronomers as they got their fair share of neutron star-related revelations during the next decade.
Important discoveries especially to study the nature of accretion, how matter infalls to a compact objects, came from the first long-duration observations done with the dutch astronomy satellite ANS.
As a direct competitor the Europeans had the mighty machinery of U.S. funded Los Alamos.
Their Vela-satellites were send to space mostly to monitor the compliance to the 1963 Partial Test Ban Treaty of nuclear weapons but they were used for science, too.
In 1975 the ANS satellite was commisioned to study possible black holes in the center of globular cluster but happened to stumble upon somthing completely different:
A short, $\sim 60$ second long X-ray flares were detected from a globular cluster NGC6624 Grindlay and Heise.\cite{GGS76}
The competing Los Alamos group found similar energetic bursts, but due to the poor angular resolution (pointing to Earth for X-rays was easy and hence no effort was put for a good spatial accuracy) they could not pin point the exact location of the sources.\cite{BCE76}
Later on, Clark et al went through the existing SAS-3 data from May 1975 and found a series of ten similar bursts from the same location, NGC6624.\cite{CJB76}
Even more retrospect, it turned out that these strange flares had been observed already in 1969 from Cen X-4\cite{BCE72} with another Vela-satellite and in 1971 with the Soviet Kosmos 428 X-ray detector\cite{BKM75}.
Their nature, however, remained elusive.

%Accretion onto a massive black hole.
%
%Pioneering work on thermonuclear instabilities on surface layers of accreting neutron stars. \cite{HvH75}
%computed stationary burning shells, found that mostly unstable
%
%X-ray bursts were due to thermonuclear flashes on the surface of accreting neutron stars.
%Laura Maraschi suggested this in early February 1976 when visiting MIT 
%X-ray bursts of nuclear origin? \cite{MC77}
%Gamma-ray bursts from thermonuclear explosions on neutron stars? \cite{WT76}
%What are they? Answer came when optical counter part of Cen X-4 \cite{vPV80} and Aql X-1 \cite{TCB78} was observed.
Pioneering theoretical work on thermonuclear instabilities on the surface layers of accreting neutron stars was initiated by Hansen and van Horn in 1975.\cite{HvH75}
They constructed stationary burning shells to lay on top of neutron stars but in contrast found out that most of them were actually unstable.
Unstable here might not give the full meaning to the physical issue though:
Such a layer on top of neutron star burning uncontrollably meant a spectacular firework.
Shortly after the Los Alamos results came in, an italian astrophysicist Laura Maraschi, while visiting MIT in February 1976 was able to connect the dots and speculated that these recently observed X-ray bursts were due to thermonuclear flashes on the surface of accreting neutron stars.\cite{MC77, Lewin93}
Similar conclusion was done by Woosley and Taam in their 1976 paper titled ``Gamma-ray bursts from thermonuclear explosions on neutron stars''.
Observational evidence soon followed when van Parajids et al and Thorstensen et al were independently able to optically resolve the companions of two known bursting sources Cen X-4\cite{vPV80} and Aql X-1\cite{TCB78}.
Not only did these observations confirm that there is a companion star close by, but that it must be such a close to the neutron star that accretion, i.e., a constant flow of new fuel for the explosions, must exist.


%\red{modern discoveries}
%The first millisecond pulsar was found in 1982 at Arecibo by Backer’s team
%
%The theory of magnetars was proposed in 1992 by Robert Duncan, Christopher Thompson and Bohdan Paczynski to explain Soft Gamma Repeaters (SGR). SGRs are repeated sources of X and $\gamma$-ray bursts. \cite{DT92}
%
%The first such object called SGR 0525-66 was discovered.
%A very intense gamma-ray burst was detected on March 5, 1979 by two Soviet satellites Venera 11 and Venera 12.\cite{MGI79}
%Superbursts in 2000\cite{CHK00}
All of the aforementioned discoveries were, however, nothing but a prelude to what was discovered on the years to follow.
We will end this short historical review by listing some of the most important more modern findings.
A big revelation came in 1979 when a very intense burst of gamma-rays was detected by two Soviet satellites Venera 11 and Venera 12.\cite{MGI79}
Later dubbed as Soft Gamma Repeaters (SGRs), their energy source remained mysterious for decades.
Theoretical breakthrough came in 1992 when Robert Duncan, Chistropher Thompson, and Bohdan Paczynski showed that the bursts could originate from a neutron star with a magnetic field $100$ to $1000$ times more powerful then what was previously known.\cite{DT92}
Today these neutron stars are more commonly known as magnetars, a subclass of young neutron stars where the initial magnetic field has been amplified by some delicate dynamo processes during the supernova explosion.
Another surprise came in 1982, when a team lead by Backer changed on how we look at pulsars when they found, using the world's largest radiotelescope in Arecibo, a pulsar spinning $641$ times per second.\cite{BKH82}
This new neutron star was dubbed as a millisecond pulsar and unlike its predesessors, we now know that instead of slowly decreasing in spin, it belongs to a class of old pulsars that have been spun up by the accretion.
In 2000, our understanding of the thermonuclear X-ray bursts was also changed when Cornelisse observed a very long, not minutes but hours long, burst from a neutron normally exhibiting regular bursts.\cite{CHK00}
These were then dubbed as superbursts, in contrast to the normal burst.
The reason in the difference is, we think, because of the burning material:
normal bursts use hydrogen and helium as their fuel but superbursts can devour a more rare carbon shell in a matter of hours if the conditions are just right.




\section{From first principles to a neutrons star}

\subsection{Background: Sun and stars}
Let us first see, what can we learn from neutron stars using simple estimates and conservation laws.
Neutrons star are born from a death of a normal star.%
\mnote{Stellar orders of magnitude}
Most familiar such a star is our Sun \Ten{1.496}{13} \cm from us.%
\footnote{Throughout this thesis we will typically present our quantities only up to some fixed precision instead of the full litany of numbers.
We will also adopt the centimeter-gram-second (cgs) unit system instead of the (maybe) more common SI-system. 
Such a selection is sure to disappoint some, but try to endure.
}
With a mass of $\Msun = \Ten{1.99}{33} \g$ and radius of $R_{\odot} = \Ten{6.96}{10} \cm$, our Sun gives us an idea of the typical stellar size scales.
Curiously, these numbers also means that the mean density of the Sun is $\rho_{\odot} \approx 1.41 \gcm$, a mere $1.4\times$ the density of the water.

Like all stars, our Sun is held together by the inward pulling gravity.
Gravity does not prefer any direction more than some other and so a spherical object is expected to form.
In addition to the inward-facing force, an outward-facing force is needed to balance the system.
For normal stars this force is originating from the thermal pressure.

We observe stars in the night sky because they shine on us.
This radiation, and also the origin of the thermal pressure, is from the thermonuclear fusion burning inside the star.
\textit{Thermo} here refers to the temperature and heat, \textit{nuclear} to the atomic nuclei, and \textit{fusion} to a process where elements are fused together.
During the thermonuclear fusion process, the star's core fuses light elements such as protons into heavier ones like helium.
Mass of two protons is less than a mass of one helium atom.
This mass difference between the start and the end results is then transferred into energy in accordance to the Einstein's famous $E = mc^2$ formula.
A whole sequence of such fusion processes takes place inside the star where lighter elements are merged together to build heavier and heavier elements.
This energy release from the mass-to-energy conversion will then give the star a sufficient thermal pressure support to keep it from not collapsing under the relentless gravity.

The fusion of elements does not continue forever.
In the beginning, two protons collide two form a helium.
In the next stage, three helium nuclei collide to form a carbon, and so on, until iron is created.
Production of iron marks the end of the possible fusion chain because fusion of two iron cores does not release energy anymore. 
On the opposite, it requires external energy source to take place.%
\footnote{This opens up another possibility of creating energy by splitting heavy elements, an inverse process to what is described here. Such a process is called fission and is familiarly taken advantage of in earthly nuclear power plants.}
This iron produced, will then sink to the center of star forming a dead core without any energy output.

Like all big furnaces, at some point the stars will run out of fuel to burn.
What is then left behind is a inner core of iron with a subsequent onion-like layers of lighter and lighter elements.
Crucial question to ask next is: 
What is supporting this iron core now that the thermal pressure from the fusion processes are lost?
To answer this, we need to look inside the iron atom:
Iron nucleus, consisting of $26$ protons and $32$ neutrons, and surrounded by $26$ electrons, repulses its neighbors because of the negatively charged electron cloud does not want to get in touch with its neighbors flying next to it in the iron-atom lattice.
\footnote{More precise consideration shows that nickel ($^{56}_{28}\mathrm{Ni}$) is actually thermodynamically more favored in the core because of the lack of neutrons needed to synthesize iron-58 ($^{58}_{26}\mathrm{Fe}$).
Underlying idea presented here, however, remains the same.
}
It is this antisocial avoidance of neighboring particles, originating from the electric charge repulsion, that will then give the internal support for the iron core not to collapse under its own gravitational pull.
If enough iron is build up during the lifetime of the star so that even this repulsion force is not enough, we can continue our thought experiment and ask, again, what will follow?
This was the question that led the scientists like Chandrasekhar to the realization of degenerate matter and white dwarf stars in the 1920s.

\subsection{White dwarfs and quantum mechanics}
The answer originates from the elusive quantum mechanics.
When the atoms inside the matter are packed close enough to each other, we need to apply wave-like characteristics for them, instead of the classical point-like thinking.
Because of their smaller mass, the electrons orbiting the nuclei enter the realm of quantum mechanics first.
A freely moving electron confined into a small enough space because of the surrounding neighbors will start to attain only some fixed values of momenta.
In physics we speak about quantization of energy levels.
The reason is similar to a vibrating string of a guitar: the string fixed from the both ends can only vibrate on some specific wave modes that are set by its length.
Additional complication for the electrons is set by the Pauli exclusion principle that forbids more than one electron to occupy a same wave mode or a quantum state inside the same region.%
\footnote{\red{Explain Pauli repulsion for a layman}}
This gives rise to a degeneracy pressure as electrons fill their quantum states from the lowest to the highest, and can, hence, not be packed any more closely.
A star held together by this degeneracy pressure of its electrons is known as a white dwarf.
From this setup it only takes a short step into realizing the existence of neutron stars, because we can, again, push forward and ask what next?%
\footnote{The reader is reassured that the chain of thermal pressure $\rightarrow$ charge repulsion $\rightarrow$ electron degeneracy will next come to a halt as the final neutron degeneracy is really the final possible supporting force in Nature. Maybe excluding the quark matter, tough\ldots}


\subsection{Neutron stars at last}
What if at some point even these quantum effects of the electrons are not enough to support the star?
One does not need to worry because after the lightweight electrons have given all they can, it is the heavy neutrons that slowly start to enter the quantum mechanical realm.
In practice, the matter will turn into a one big team of neutrons because when the positive ($+$) central proton and the surrounding negative ($-$) electron come in contact, a neutral neutron is created.%
\footnote{In reality the beta decay formula is $e^- + p = n + \nu_e$, where the additional electron neutrino is needed to not only the charge neutrality but also the quark color neutrality.}
The degeneracy pressure of such a neutron porridge is multiple orders of magnitude larger than what the electrons can offer yielding an ultimate solution to the pressure support problem.

Let us consider the consequences of this thought play.
More detailed calculations show that the resulting iron-core sitting at the center of the star is weighting at a maximum of around $\sim \Msun$.%
\footnote{This is quite reasonably sounding assumption taken into account that the stars that explode are of around $\sim10\Msun$ and we certainly do not expect everything to fall into the core.
}
Hence, there are $M_{\mathrm{core}} / m_{\mathrm{atom}} \sim \Msun / \mproton \approx \Ten{1.99}{33} \g / \Ten{1.67}{-24} \g \sim 10^{57}$ atoms trapped inside the core.
\footnote{
    The mass of the atom, $m_{\mathrm{atom}} = \mproton + \me$, is approximated (to an excellent accuracy) by just considering the central nuclei alone as the electron mass $\me \approx \Ten{9.11}{-28} \g$ is negligible in comparison to the proton mass.
}
Here we are already considering not iron atoms but pure hydrogen atoms only to simplify the presentation.
Working backwards from these numbers, we can estimate the size of the compressed core.
Using a typical radius for the nuclei of $r_{\mathrm{n}} \approx \Ten{1.25}{-13}\cm$, we would expect these particles to form an object of around $R \sim (10^{57})^{\frac{1}{3}} \times \Ten{1.25}{-13}\cm \sim 10^6 \cm$.
Hence, we have ended up in a star consisting of only neutrons, with a size of $\sim 10\km$ and mass of $\sim 1\Msun$, a neutron star!

By considering simple order-of-magnitude estimates we have now ended up characterizing the dimensions of a typical neutron star.\mnote{Density}
A canonical neutron star is often taken to have $R=10\km$ and $M=1.4\Msun$, so let us also adopt those numbers for the following considerations.
Such dimensions give us an impressive mean density of $\rho \sim \Ten{7}{14} \gcm$.
In comparison, for a typical nucleon (such as a neutron or a proton) we had $m_{p} \approx m_{n} \approx \Ten{1.67}{-24} \g$ and $r_{n} \approx \Ten{1.25}{-13} \cm$, yielding us a nuclear density of $\nsat \approx \Ten{2}{14} \gcm$.
Not surprisingly, the densities are of similar magnitude.
However, when comparing to our every-day matter, the difference is huge, almost $14$ orders of magnitude:
A cubic centimeter of water weights $1\g$ whereas a same volume of neutron star matter would weight $100\,000\,000\,000\,000\g$ or $100$ million metric tons.


Matter compressed to such a small volume has an extreme impact even on the surrounding spacetime.\mnote{Escape velocity}
Let us try to estimate, again, the order-of-magnitude of these effects by considering the escape velocity --- a velocity needed to escape the local gravitational pull of an object.
For us, on top of Earth, it turns out to be $v_{\mathrm{\earth}} = \sqrt{2G M_{\earth} / R_{\earth}} = \Ten{1.12}{6} \cms$, for $M_{\earth} = \Ten{5.97}{27} \g$ and $R_{\earth} = \Ten{6.37}{8} \cm$.
Similarly, for the Sun it is $v_{\odot} = \Ten{6.18}{7} \cms$, or $0.002 \times$ the speed of light.
On the other hand, for a neutron star we obtain $v_{\mathrm{NS}} = \Ten{1.93}{10} \cms$, which is already about half of the speed of light!
Hence, relativistic effects become crucial to take into account when considering neutron stars.


Let us next think about the possible spins rates that a neutron star can have.\mnote{Spin}
For our Sun, it takes $25.5$ days or about one month to revolve around itself, corresponding to a spin rate of $\Ten{4.5}{-7}\Hz$.
When compressed to a dimensions that of a neutron star the radius changes by a factor of $R_{\odot} / R_{\mathrm{NS}} \approx \Ten{6.96}{10}\cm / 10^6 \cm \sim \Ten{7}{4}$.
It is important to notice that when a rotating object collapses, it preserves it's angular momentum not the spin rate.
Similar to a ice-figure skater pulling her arms inwards while spinning, we observe an increase in the spin in order to preserve the angular momentum.
As the rotational inertia increases as a square from the distance from the axis, our Sun, when compressed to a neutron star, would obtain a spin of $\Ten{4.5}{-7}\Hz \times (\Ten{7}{4})^2 \sim \Ten{2}{3} \Hz$.
In reality, the young proto neutron star quickly slows down after its birth, and we are left with spins of around $10^2$ to $10^3 \Hz$, still about one revolution per $1$ to $10$ milliseconds.


%716 Hz from Hessels et al. (2006), 
%    http://dx.doi.org/10.1126/science.1123430 .
%1/11.79 Hz from Rim and Kaspi (2014),
%    http://dx.doi.org/10.1088/0004-637X/784/1/37 .


A final characteristic observable, we can try and estimate is the magnetic field.\mnote{$B$-field}
Here we can follow a similar chain of reasoning as with the spin and start from a typical values for our Sun.
For the Sun, the slow rotation gives rise to a dynamo process that produces a magnetic field of around $B_{\odot} \approx 1\Gauss$.%
\footnote{A typical refrigerator magnet is about $50\times$ stronger with a magnetic field of $50\Gauss$.}
When considering magnetic field, it is the magnetic flux through the surface that conserves, hence we expect the field to scale also as square the radius.
Using the same compression ratio of \Ten{7}{4} for the radii, we then obtain $B_{\mathrm{NS}} \approx 1 \times (\Ten{7}{4})^2 \Gauss \sim 10^{10} \Gauss$.
Comparing this to the strongest non-destructive magnet on Earth of $10^6\Gauss$, we start to grasp the level of energetics that the neutron stars have to offer.


%Magnetic fields in pulsars $10^{11} - 10^{13}$ G \cite{MHT05}
%in binaries accretion screens b field \cite{CZB01}

%Maximum from virial theorem $B \sim 10^{18} - 10^{19}\Gauss$.
%After that magnetic energy $R^3 B^2 / 6$ is grater than graviational binding energy $3 G M^2 / 5 R$.\cite{CF53,ST83,LS91}

%B=2e15 for SGR 1806-20 from Woods et al. (2006),
%        http://dx.doi.org/10.1086/507459 .
%B=10^7 is estimate from P-Pdot diagrams, e.g.
%        from arxiv.org/abs/1103.4538 
%


\section{Connection to this thesis}
In this thesis we study many aspects of the neutron stars from their environments to the internal processes to help us understand their Nature better.
In the end, our main goal is to use this plethora of information to better constrain the behavior of the ultra-dense matter inside the core.




