\chapter{Introduction} 
\chapterimage[width=6cm]{wordcloud/chap1.png}
Neutrons stars are curious objects encompassing many still unsolved problems of modern physics and astrophysics.
Their unique nature makes them ideal laboratories for many of the most energetic phenomena in space.
Born from the ashes of a supernova, they begin their life only when some other normal star fades away and dies in a spectacular supernova explosion.%
\footnote{ {For the pedantic ones, we specify that a supernova is actually an implosion followed by a subsequent explosion.}}
From there on, they continue their life by devouring the surrounding interstellar matter or an unlucky companion star floating next to them.
It is not the impressive $\sim 10^{30}$ kilos they weight but the mere $\sim 10 \km$ in radius sphere that they encapsulate this material into that is then able to bend the spacetime itself.
Such an impressive feat rewards them with a categorization into to a group called \textit{compact objects}.
These strange objects have been under scientific scrutiny almost for nine decades now.
Still, some of the most fundamental questions remain:
What are they made of?
How big are they?
In which ways are they actually visible to us?

In the next few sections, we will start by building our intuition of these peculiar objects habiting the space around us.



\section{Short history: from imagination into a reality}
Anticipation by theorist before the second World War. \cite{Landau32}
\begin{displayquote}[Landau]
...the density of matter becomes so great that atomic nuclei come in close contact, forming one gigantic nucleus.
\end{displayquote}

Blaablaa\cite{Baade34a}
\begin{displayquote}[Baade och Zwicky]
With all reserve we advance the view that a super-nova represents the transition of an ordinary star into a neutron star, consisting mainly of neutrons. Such a star may possess a very small radius and an extremely high density.
\end{displayquote}

Work of imagination from weird theorists.
On July 1967 everything changed because of Jocelyn Bell, a graduate student of A. Hewish in Mullard Radio Astronomy Observatory.

She discovered that the signal was pulsing with great regularity, at a rate of about one pulse per second. 
Temporarily dubbed "Little Green Man 1" (LGM-1) the source (now known as PSR B1919+21) was identified after several years as a rapidly rotating neutron star.
Later published in Nature 1968 by Hewish. \cite{Hewish68}



\section{From first principles to a neutrons star}
Let us first see, what can we learn from neutron stars using simple estimates and conservation laws.

Suppose a neutron star is, like any normal star, a blob of gas held together by the inwards pulling gravity.
Gravity does not prefer any direction more than some other and so a stable end-result is an isotropic configuration.
A pure inward pulling force is, of course, not enough so we also need a countering outward-facing force to resist the compression of the material.
As an first approximation, there is no need to assume that this force would have any preferred direction either.
Hence, our expected outcome is a sphere held together by the gravity originating from the mass $M$ of the matter itself.
Let us, for a while forget the exact origin and nature of the compression-resisting force and see what can we learn solely from the current information only.


Neutrons star are born from the death of a normal star.%
\mnote{Orders of magnitude}
Most familiar such a star is our Sun \Ten{1.496}{13} \cm from us.%
\footnote{Throughout this thesis we will typically present our quantities only up to some fixed precision instead of the full litany of numbers.
We will also adopt the centimeter-gram-second (cgs) unit system instead of the (maybe) more common SI-system. 
Such a selection is sure to disappoint some, but try to endure.
}
With a mass of $\Msun = \Ten{1.99}{33} \g$ and radius of $R_{\odot} = \Ten{6.96}{10} \cm$, our Sun then introduces us some typical stellar dimensions.
Curiously, this means that the mean density of the Sun is $\rho_{\odot} \approx 1.41 \gcm$, a mere $1.4\times$ the density of the water.
It turns out that Sun is also not as stable as one would think:
With a rotation period of about $25.5$ days it then takes Sun about a month to revolve around itself.
Similar to a bicycle dynamo hub, this rotation also gives rise to a detectable surface magnetic field of $B_{\odot} \approx 1 \Gauss$.%
\footnote{A typical refrigerator magnet is about $50\times$ stronger with a magnetic field of $50\Gauss$.}

A typical neutron star, on the other hand, weights about $M \sim 1.5\Msun$ but extends only up to $R \sim 10\km$.
Such dimensions give us an impressive mean density of $\rho \sim \Ten{7}{14} \gcm$.
In comparison, a typical nucleon (such as a neutron) weights about $\Ten{1.67}{-24} \g$ and has a radius of about $\Ten{1.25}{-13} \cm$, yielding us a nuclear density of $\nsat \approx \Ten{2}{14} \gcm$.
Hence, even the mean density inside the star is already on the same order of magnitude as the internal density inside nuclei.
This suggests us that the composition inside a neutron star is not our typical every-day matter.


\section{What do we actually know?}


%B=2e15 for SGR 1806-20 from Woods et al. (2006),
%        http://dx.doi.org/10.1086/507459 .
%B=10^7 is estimate from P-Pdot diagrams, e.g.
%        from arxiv.org/abs/1103.4538 
%
%
%716 Hz from Hessels et al. (2006), 
%    http://dx.doi.org/10.1126/science.1123430 .
%1/11.79 Hz from Rim and Kaspi (2014),
%    http://dx.doi.org/10.1088/0004-637X/784/1/37 .





\section{More advanced considerations...}

\subsection{Why neutrons then?}
Let us first consider ideal gas of degenerate electron-proton-neutron plasma.
In a degenerate plasma all the quantum states are filled up all the way to the Fermi energy.
It is the Pauli exclusion principle that then prevents occupying all of these already taken quantum states.
Normal beta-decay mode for the neutrons, on the other hand, is $n \rightarrow p + e^{-} + \bar{\nu_{e}}$, that describes the possible path of how a neutron $n$ will decay into a proton $p$, electron $e^{-}$, and electron neutrino $\bar{\nu_{e}}$.
Such a decay is, however, blocked because there is no room for an emission of an extra electron $e^{-}$ or a proton $p$.\cite[see e.g.][]{Phillips94}

Let us then only focus on the decay of the most energetic neutrons with an energy equal to the Fermi energy $\ef(n)$.
Co-existence of neutrons, protons, and electrons is then guaranteed (at zero temperature) if 
\be
\ef(n) = \ef(p) + \ef(e^{-}).
\ee
Fermi momentum of a particle is related to its concentration via
\be
p_{\mathrm{F}} = \left( \frac{3n}{8\pi} \right)^{1/3} h,
\ee
where $n$ is the number density, and $h$ the Planck constant.
Massive neutrons and protons are to a good approximation non-relativistic up to a densities of $\nsat$, and hence energy is simply a sum of their rest mass energy and kinetic energy
\be
\ef(n) \approx m_n c^2 + \frac{p_{\mathrm{F}}(n)^2}{2 m_n },
\ee
and
\be
\ef(p) \approx m_p c^2 + \frac{p_{\mathrm{F}}(p)^2}{2 m_p }.
\ee
Electrons, on the other hand, are already ultra-relativistic, and so
\be
\ef(e^{-}) \approx p_{\mathrm{F}}(e^{-}) c^2.
\ee

\red{
Also note that $n_p = n_e$, as the star is electrically neutral.
From this we find relation of the $n_n/n_p \sim 1/200$ by taking into account the rest mass difference $m_p - m_n = 2.6 \mathrm{MeV}\,c^2$ at $\rho \sim \nsat$.
}
Thus, we conclude that the matter inside is neutron rich.



\subsection{Tolman-Volkoff-Oppenheimer equations}
Newtonian pressure gradient needed to oppose the gravity is
\be
\frac{dP}{dr} = -\frac{G M \rho}{r^2}.
\ee
Taking into account the general relativistic corrections we get
\be
\frac{dP}{dr} = 
    -\frac{G M \rho}{r^2} \times 
    \frac{ (1 + P/\rho c^2)(1+4\pi r^3 P/m c^2) }
    {1-2 G m /r c^2 }.
\ee
Difference originates from the source of gravity:
in the Newtonian case it is the mass $m$, whereas in the General relativity it is the energy momentum tensor that depend both on the energy density and the pressure.
As a result, energy and pressure give rise to a gravitational fields.

It has an important consequence to the stability of neutron stars:
Successive increase in the pressure to counter the gravity is ultimately self-defeating.

\red{
Solution for a constant density $\rho_0$ gives
\be
P(r) = G \frac{2\pi}{3} \rho_0^2 (R^2 - r^2)
\ee
whereas the GR gives
\be
P(r) = \rho_0^2 c^2 \left[ \frac{ (1-u \left(\frac{r}{R} \right)^2 )^{1/2}
                        - (1-u)^{1/2} }{
                        3(1-u)^{1/2} - (1-u \left( \frac{r}{R} \right)^2 )^{1/2} } \right],
\ee
where $u = 2GM/Rc^2$.
}


\section{Equation of state}

Often means dependency between $P$ and $\rho$. 
Or sometimes the associated energy density $\epsilon = \rho c^2$.
Also depends on $T$ but composed mainly on strongly degenerate fermions so so temperature dependency is negligible.

Bulk property of the sea of fermions.

Eos for $\rho > \nsat$ can not be produced in laboratory.
Can not be calculated because of the lack of precise many-body theory of strongly interacting particles.


Baryon mass $M_b$ that is sum of baryon masses.
Gravitational mass $M$ that is $M_b$ from where the gravitational binding energy is subtracted. \cite{Zwicky38}


\section{Atmosphere}
Thin layer of plasma
From centimeters in hot to millimeters in cold 
Zavlin \& Pavlov 2002

Where spectrum or thermal electromagnetic radiation is formed.
Spectrum, beaming and polarization of emerging radiation can be determined from radiation transfer problem in atmospheric layers.

Contains information on the parameters of the surface:
effective temperature
surface gravity
chemical composition
geometry of the system
mass and radii.


Eddington limit of where radiation force exceeds the gravitational one.


\section{Crust}

\begin{figure}
\centering
\includegraphics[width=16cm]{figs/nstar-plot/crust_plot_wide.png}
\caption{\label{fig:crust}
Molecular simulation of the crust.
Figure adapted from \url{https://github.com/awsteiner/nstar-plot}.
}
\end{figure}

Outer crust

From atmosphere to $\rho_ND \sim \Ten{4}{11} \gcm$.
In thickness some hundred meters.
Non degenerate electron gas
Ultra-relativistic electron gas $\rho > 10^6 \gcm$.
Pressure provided by electrons here.

In deeper layers ions form a strongly coupled Coulomb system (liquid or solid).
Hence, crust.
Fermi energy grows with increasing $\rho$.
Induces $\beta$ captures and enriches nuclei with neutrons.
At the base neutrons start to drip out from nuclei.


Inner crust
About one kilometer thick.
Density from $\rho \sim \rho_{ND}$ (at upper boundary) to $\sim 0.5 \nsat$ at the base.
Matter consists of electrons, free neutrons $n$ and neutron-rich atomic nuclei.
Fraction of free $n$ grow with $\rho$.

Finally, nuclei disappear at the crust-core interface.

\section{Core}

Outer core.
Density ranges $0.5 \nsat < \rho < 2\nsat$.
Several kilometers.
Neutrons with several per cent admixture of protons $p$ and electrons $e^{-}$.
Strongly degenerate.
Electrons form almost ideal Fermi gas.
Neutrons and protons, interacting via nuclear forces, constitute a strongly interacting Fermi liquid.

Inner core.
Where $\rho > 2\nsat$.
Central density can be around $(10-15)\nsat$.
Very model dependent.
Main problem.




