\chapter{Probing the ultra-dense matter}
\chapterimage[width=15cm]{wordcloud/chap4b.png}

The main motive for this thesis was to set constraints for the ultra-dense equation of state.
Instead of starting from the nuclear physics that works on the smallest scales, we use astrophysical observations to study large-scale ``global'' aspects of neutron stars.
It is then possible to make a step back to the nuclear physics because the size of a compact star is strongly coupled to the composition of its core.

Looking from the astrophysical point of view, it is the size of the neutron star that will define many of its observable features.
One of the most important characteristics is the compactness of the object that will then define the exact shape of the spacetime surrounding it.
The strongly curved spacetime, in turn, influences many of the phenomena occurring in the close vicinity of the star and will also leave its distinct imprints on the observations.

The physical phenomena behind the observable features on the other hand, are often highly energetic, otherwise they would not be seen by distant observers, such as us.
It is these highly energetic physical processes that will then render the neutron stars visible to us, and that at the same time carry a plethora of information from the surroundings of where they originated from.
This gives birth to a beautiful cosmic connection where the delicate and unattainable nuclear physics of the ultra-dense matter is coupled to vigorous astrophysical phenomena that we can observe.
The caveat here is that the astrophysical processes are often messy and poorly understood.
Hence, a thorough understanding of both, the nature of the observed phenomena and how it exactly couples to the nuclear physics, is needed.

In this thesis, we will focus on extracting the information from the X-ray bursts.
In theory, this method of using the X-ray bursts to probe the neutron star interiors is robust as we can theoretically model the characteristics of the emerging radiation and these models can be applied to describe the data that we see. 
In practice, however, caution is needed when applying the models as the environment and the astrophysics near the neutron star play a huge role.


In this final chapter, we will lay out the basics of how observing the burst cooling it is possible to set constraints on the size of the emitting area, and in the end, the radius of the neutron star.
We will also summarize the content of the articles in this thesis, and discuss discuss our work where we try to understand not only the complex role of the astrophysical surroundings but in the end, the composition of the core.


\section{Measuring the size of the bursting source}

Even though the bursts characteristics change from one burst to another as we saw in Sect. \red{XXX}, the cooling appears to obey some common trends.
This means that as long as we have some kind of an energy injection deep below the neutron star's atmosphere, the energy will radiate out and the uppermost layers of the star will then shape into a similar cooling curve, independent of the actual details of the injection.
If we are then able to model the atmosphere and the processes therein, we can then use the bursts as a probes for the neutron star interiors.




The big caveat here is the surroundings.
We have seen that the astrophysical environment of neutron stars can be very active and lively.
In the general picture, we have the accretion as an energy source and the disk to dissipate this energy.
The disk is, however, not a simple geometrically thin steady layer but can have complex inner flow.
On the other hand, if the disk does extend all the way down to the star, an additional complication originates from the boundary or spreading layer that can not only cover the star but also radiate on its own.
These are some of the complications that we face when trying to analyze our neutron star observations, as after all when trying to constrain the mass and radius of the star we must make sure that it is actually the star that we are looking.




\section{The author's contribution to the publications} 

\subsubsection*{Paper I.}
The author of the thesis made contributions to the manuscript, reduced and analyzed the observational X-ray data, and contributed to the scientific discussions related to the manuscript.


\subsubsection*{Paper II.}
The author participated in the reduction and analysis of the observational data, made significant contributions to the development of the data reduction software, and helped in the preparation of the manuscript.


\subsubsection*{Paper III.}
The author contributed to the main idea of the paper, independently redesigned the neutron star atmosphere code used for the calculations, and implemented new physical processes to this numerical framework. 
The author also prepared most of the manuscript.


\subsubsection*{Paper IV.}
The author independently designed the Bayesian fitting framework for the cooling tail method, reduced and analyzed the X-ray observations, and finally led the equation of state modeling from the observations. 
The author also prepared the manuscript.


\subsubsection*{Paper V.}
The author contributed to the main idea of this research and was responsible of the atmosphere modeling of the observations. The Bayesian atmosphere spectral model fitting framework was also independently designed by the author. 
Author also made significant contributions to the manuscript.


\subsubsection*{Paper VI.}
Author helped in designing the fitting framework, based on his own previous results, and contributed to the scientific and statistical discussions of the paper. 
The author also contributed to the manuscript.


\subsubsection*{Paper VII.}
In this paper, the author proposed the usage of the dynamic power-law method and co-supervised the project which was originally based on the Master's thesis of J. Kuuttila. 
Author also made significant contributions to the manuscript.


\subsubsection*{Paper VIII.}
The author of the thesis took part in the discussion of the theoretical explanation for the obtained observational results and contributed significantly to the statistical analysis of data. 
The author also contributed to the manuscript.


\subsubsection*{Paper IX.}
The author independently proposed the idea of applying the split-Hamilton method to the ray tracing problem of photons, derived the theoretical framework and all the related formulae, designed the numerical code, and prepared most of the manuscript.


\subsubsection*{Paper X.}
The author independently designed the hierarchical Bayesian fitting framework, implemented it into a code together with M.C. Miller and A.W. Steiner, analyzed the data, and, finally, prepared most of the manuscript together with M.C. Miller.


\subsubsection*{Paper XI.}
In this paper, the author took part in the scientific discussion of the results, helped in the statistical analysis and contributed to the manuscript.




