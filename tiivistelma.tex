
\begin{otherlanguage}{finnish}
\vspace{2.0cm}
\chapter*{Tiivistelmä}


Neutronitähdet ovat universumimme tiheimpiä tähtiä. 
Niiden sisältämän erittäin tiheän kylmän aineen tilanyhtälö ja tarkka käyttäytyminen ovat kuitenkin vielä tuntemattomia. 
Täs\-sä väi\-tös\-kirjassa näy\-tän kuinka kaukaisenkin neutronitäh\-den koko voidaan mitata hyö\-dyn\-tä\-en niin kutsuttujen rönt\-genpurkausten lähet\-tämää sä\-tei\-lyä. 
Röntenpurkaukset saavat alkunsa termisestä fuusioreaktiosta joka tuottaa valtaisan räjähdyksen tähden pintakerroksissa. 
Mittaamalla ja mallintamalla näistä purkauksista syntyvää säteilyä, saamme tietoa neutronitähtien sisältämän aineen käyttäytymisestä ja siten myös kylmän tiheän aineen tilanyhtälöstä.

Mittaukset tehdään vertaamalla neutronitähtien pinnalta alkunsa saavaa säteilyä teoreettisiin ilmakehämalleihin jotka ennustavat kuinka pinnan tulisi jääh\-tyä purkausten jäl\-keen. 
Tämän takia tarvitsemme tarkkoja malleja säteilyn kulusta ilmakehässä. 
En\-simmäi\-sessä osassa väi\-tös\-kir\-jaani olen tutkinut kuinka ilmakehässä olevat raskaat fuusioreaktioissa syntyneet alkuaineet vaikuttavat tämän säteilyn etenemiseen ilmakehän plasmassa. 
Tämä auttaa meitä ymmärtämään ja tulkitsemaan myös röntgenpurkauksista tehtyjä havaintoja. 
Lisäksi olen näyttänyt kuinka havaittu säteily muuttuu, kun se saa alkunsa erittäin nopeasti pyörivästä ja navoiltaan litistyneestä neutronitähdestä.

Tarkkojen ilmakehämallien lisäksi meidän täytyy myös ymmärtää mitä neutronitähden ympärillä tapahtuu. 
Väitöskirjani toisessa osassa tutkin kuinka ympäristö voi vaikuttaa herkkiin tähden säteen mittauksiin, koska joskus neutronitähden pinnalle putoava materia voi häiritä mittauksia. 
Tärkein löydöksemme on, että säteen luotettavaan mittaamiseen voidaan käyttää vain sellaisia purkauksia, jotka tapahtuvat kun putoavaa materiaa on erittäin vähän.

Kun edellä mainitut seikat huomioidaan on mahdollista mitata neutronitähden koko, etäisyys, ja ilmakehän koostumus vertaamalla oikeiden, havaittujen rönt\-gen\-pur\-kaus\-ten jäähtymistä mallien ennusteisiin. 
Viimeisessä osassa väitöskirjaani olen tutkinut kolmen eri neutronitähden röntgenpurkausten säteilyä.
Kyseiset neutronitähdet sijaitsevat kaksois\-täh\-ti\-jär\-jes\-tel\-missä 4U 1702$-$429, 4U 1724$-$307, ja SAX J1810.8$-$260. 
Kyseisten neutronitähtien säde on mittauksieni mukaan $10.9$ ja $12.4$ km välillä ($68\%$ luottamustaso). 
Uusien ilmakehämallien avulla olemme myös todistaneet, että kaksois\-täh\-ti\-jär\-jes\-tel\-mässä HETE J1900.1$-$2455 sijaitsevan neutronitähden pintakerrokset sisältävät fuusioreaktion aikana syntyneitä raskaita alkuaineita. 
Kehitin myös uudenlaisen Bayesilaisen metodin, jossa ilmakehämalleja voidaan sovittaa suoraan röntgenpurkauksista tehtyihin havaintoihin. 
Tätä metodia käyttäen mittasin 4U 1724$-$429:ssä sijaitsevan neutronitähden sä\-teek\-si $R=12.4 \pm 0.4\km$ ($68\%$ luottamustaso). 
Nämä uudet tulokset ovat sopusoinnussa uusien ydinfysikaalisten ennusteiden kanssa. 
Lisäksi ne näyttävät kuinka astrofysikaalisia mittauksia voidaan käyttää apuna ydinfysiikan tutkimuksessa.

\end{otherlanguage}
