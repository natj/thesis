\chapter{Physics of neutron star interiors} 
\chapterimage[width=15cm]{wordcloud/chap2b.png}



\section{Equation of state}

Often means dependency between $P$ and $\rho$. 
Or sometimes the associated energy density $\epsilon = \rho c^2$.
Also depends on $T$ but composed mainly on strongly degenerate fermions so so temperature dependency is negligible.

Bulk property of the sea of fermions.

Eos for $\rho > \nsat$ can not be produced in laboratory.
Can not be calculated because of the lack of precise many-body theory of strongly interacting particles.


Baryon mass $M_b$ that is sum of baryon masses.
Gravitational mass $M$ that is $M_b$ from where the gravitational binding energy is subtracted. \cite{Zwicky38}


\section{Atmosphere}
Thin layer of plasma
From centimeters in hot to millimeters in cold 
Zavlin \& Pavlov 2002

Where spectrum or thermal electromagnetic radiation is formed.
Spectrum, beaming and polarization of emerging radiation can be determined from radiation transfer problem in atmospheric layers.

Contains information on the parameters of the surface:
effective temperature
surface gravity
chemical composition
geometry of the system
mass and radii.


Eddington limit of where radiation force exceeds the gravitational one.


\section{Crust}

\begin{figure}
\centering
\includegraphics[width=16cm]{figs/nstar-plot/crust_plot_wide.png}
\caption{\label{fig:crust}
Molecular simulation of the crust.
Figure adapted from \url{https://github.com/awsteiner/nstar-plot}.
}
\end{figure}

Outer crust

From atmosphere to $\rho_ND \sim \Ten{4}{11} \gcm$.
In thickness some hundred meters.
Non degenerate electron gas
Ultra-relativistic electron gas $\rho > 10^6 \gcm$.
Pressure provided by electrons here.

Here the equation of state is decribed by the relativistic degenerate electron gas.
Physics behind this are quite simple and we repeat the calculations here to give the reader ideas about what are the most important physical processes.
The result also bears some historical value.

We have reached densities where the EoS is dominated mainly by the electrons, hence it is characterized by the electron number density $n_{\mathrm{e}}$ and temperature $T_{\mathrm{e}}$ (hereafter just $T$ in this section).
Moving on from an ideal plasma, we can start by introducing corrections produced by the closely packed charges.
In practice we can use the so-called ion-sphere model to describe our Coulomb liquid of ions.
We now assume that our ions are emerged into a sea of rigid electron background that takes care of the charge neutrality.
Let us begin by defining a so-called electron sphere radius
\be
\erad = \left( \frac{4\pi \nel}{3} \right)^{-1/3}
\ee
We can also parameterize the strength between Coulomb (charge) interactions by considering a ratio of potential energy to the kinetic energy with
\be
\Ge = \frac{e^2}{\erad kT}.
\ee
Similarly, for ion with a charge number of $\Zi$, we can define the ion-sphere radius
\be
\irad = \erad \Zi^{1/3}
\ee
that encapsulates enough area to be charge neutral, when considering a static electron-induced background from $\nel$.
Ion Coulomb coupling factor is similarly
\be
\Gi = \Ge \Zi^{5/3} = \frac{ (\Zi e)^2}{\irad kT}
\ee
In the weak-coupling limit ($\Gi \ll 1$) Debye-H\"uckel results for the free energy are valid\cite{LL80}
\be
\frac{F_{\mathrm{ex}}}{V} = \frac{1}{\sqrt{3}} n_{\mathrm{i}} kT \Gamma^{3/2}
\ee
Hence, the pressure correction due to the Coulomb interactions is\cite{DeWitt96}
\be
\Pii \approx -0.3 n_{\mathrm{i}} \frac{Z^2 e^2}{\irad}.
\ee



For a degenerate system it also makes sense to present the Fermi quantities:
momentum
\be
p_{\mathrm{F}} = \hbar (3\pi^2 n_{\mathrm{e}})^{1/3}
\ee
energy
\be
\epsilon_{\mathrm{F}} = c^2 \sqrt{ ( \me c)^2 + p_{\mathrm{F}}^2 },
\ee
and temperature
\be
T_{\mathrm{F}} = \frac{\me c^2}{k} \left( \sqrt{1 + \left( \frac{ p_{\mathrm{F}} }{\me c } \right)^2 } - 1 \right)
= \Tr ( \gammar -1 ),
\ee
where we have defined a typical temperature
\be
\Tr \equiv \frac{ \me c^2 }{k} \sim \Ten{6}{9} \Kelvin
\ee
relativistic scaling factor
\be
\gammar \equiv \sqrt{1 + \xr^2},
\ee
and typical (dimensionless) momentum
\be
\xr \equiv \frac{p_{\mathrm{F}}}{\me c }.
\ee

Using these definitions, it is easy to characterize our electron gas into regions of 
\begin{itemize}
    \item non-relativistic, for which $T \ll \Tr$ and $\xr \ll 1$, 
    \item mildly-relativistic, $T \sim \Tr$ and $\xr \sim 1$,
    \item ultra-relativistic, $T \gg \Tr$ and $\xr \gg 1$,
    \item non-degenerate, $T \gg T_{\mathrm{F}}$,
    \item mildly degenerate, $T \sim T_{\mathrm{F}}$,
    \item and strongly degenerate $T \ll T_{\mathrm{F}}$.
\end{itemize}

Free energy
\be
F = (\mu  - \me c^2) \nel - \frac{2}{(2\pi \hbar)^3} \int d\vec{p} \frac{1}{3} \vec{p} \cdot \vec{v} \fFD(\epsilon)
\ee
where 
\be
\nel = \frac{2}{(2\pi\hbar)^3} \int d\vec{p} \fFD(\epsilon),
\ee
and
\be
\fFD(\epsilon, T) = \frac{1}{\exp \left( \frac{\epsilon - \mu}{kT} \right) - 1}
\ee
and
\be
\epsilon = \sqrt{ \me^2 c^4 + p^2 c^2 }
\ee

Pressure 
\be
P = -\left( \frac{ \partial F }{\partial V } \right)_{T, \{N_j \} }
\ee

Sommerfield expand free energy in powers of $T/T_{\mathrm{F}}$ 
\be
\frac{F}{V} = \frac{\me c^2}{\bar{\lambdaC^3}}  \frac{1}{8\pi^2} \left( \xr(1+2\xr^2) \gammar - \ln(\xr + \gammar) + \frac{4\pi^2}{9} \tr^2 \xr(\gammar + \gammar^{-1})  \right)  + \mathcal{O}(\tr^4)
\ee
and obtain pressure 
\be
\Peid = \frac{\Pressr}{8\pi^2} \left( \xr(1+2\xr^2) \gammar - \ln(\xr + \gammar) \right)
\ee
where again typical pressure
\be
\Pressr = \frac{\me c^2}{\lambdaC} \sim \Ten{1.4}{25} \dyncm
\ee

Hence, it takes the simple polytropic form
\be
\Peid \approx \frac{ \Pressr }{9\pi^2 \gammaad} \xr^{3\gammaad}
\ee
where the polytropic index $\gammaad = \frac{5}{3}$ for the non-relativistic $\xr \ll 1$ and $\gammaad = \frac{4}{3}$ for the ultra-relativistic case (recall also that $\xr \propto \nel^{1/3}$).


Degenerate electron gas pressure accompanied with the ion Coulomb correction will then actually give us a rather good approximation for the equation of state
\be
P(\xr) = \Peid + \Pii
\ee
this is valid in a large density range of $10^{4} < \rho < 10^{10} \gcm$.

In deeper layers ions form a strongly coupled Coulomb system (liquid or solid).
Hence, crust.
Fermi energy grows with increasing $\rho$.
Induces $\beta$ captures and enriches nuclei with neutrons.
At the base neutrons start to drip out from nuclei.


Inner crust
About one kilometer thick.
Density from $\rho \sim \rho_{ND}$ (at upper boundary) to $\sim 0.5 \nsat$ at the base.
Matter consists of electrons, free neutrons $n$ and neutron-rich atomic nuclei.
Fraction of free $n$ grow with $\rho$.

Finally, nuclei disappear at the crust-core interface.

\section{Core}

Outer core.
Density ranges $0.5 \nsat < \rho < 2\nsat$.
Several kilometers.
Neutrons with several per cent admixture of protons $p$ and electrons $e^{-}$.
Strongly degenerate.
Electrons form almost ideal Fermi gas.
Neutrons and protons, interacting via nuclear forces, constitute a strongly interacting Fermi liquid.

Inner core.
Where $\rho > 2\nsat$.
Central density can be around $(10-15)\nsat$.
Very model dependent.
Main problem.

\subsection{Why neutrons then?}
Let us first consider ideal gas of degenerate electron-proton-neutron plasma.
In a degenerate plasma all the quantum states are filled up all the way to the Fermi energy.
It is the Pauli exclusion principle that then prevents occupying all of these already taken quantum states.
Normal beta-decay mode for the neutrons, on the other hand, is $n \rightarrow p + e^{-} + \bar{\nu_{e}}$, that describes the possible path of how a neutron $n$ will decay into a proton $p$, electron $e^{-}$, and electron neutrino $\bar{\nu_{e}}$.
Such a decay is, however, blocked because there is no room for an emission of an extra electron $e^{-}$ or a proton $p$.\cite[see e.g.][]{Phillips94}

Let us then only focus on the decay of the most energetic neutrons with an energy equal to the Fermi energy $\ef(n)$.
Co-existence of neutrons, protons, and electrons is then guaranteed (at zero temperature) if 
\be
\ef(n) = \ef(p) + \ef(e^{-}).
\ee
Fermi momentum of a particle is related to its concentration via
\be
p_{\mathrm{F}} = \left( \frac{3n}{8\pi} \right)^{1/3} h,
\ee
where $n$ is the number density, and $h$ the Planck constant.
Massive neutrons and protons are to a good approximation non-relativistic up to a densities of $\nsat$, and hence energy is simply a sum of their rest mass energy and kinetic energy
\be
\ef(n) \approx m_n c^2 + \frac{p_{\mathrm{F}}(n)^2}{2 m_n },
\ee
and
\be
\ef(p) \approx m_p c^2 + \frac{p_{\mathrm{F}}(p)^2}{2 m_p }.
\ee
Electrons, on the other hand, are already ultra-relativistic, and so
\be
\ef(e^{-}) \approx p_{\mathrm{F}}(e^{-}) c^2.
\ee

\red{
Also note that $n_p = n_e$, as the star is electrically neutral.
From this we find relation of the $n_n/n_p \sim 1/200$ by taking into account the rest mass difference $m_p - m_n = 2.6 \mathrm{MeV}\,c^2$ at $\rho \sim \nsat$.
}
Thus, we conclude that the matter inside is neutron rich.


\subsection{Polytropes}
\red{Parameterize everything with polytropes.}



\section{Tolman-Volkoff-Oppenheimer equations}
Newtonian pressure gradient needed to oppose the gravity is
\be
\frac{dP}{dr} = -\frac{G M \rho}{r^2}.
\ee
Taking into account the general relativistic corrections we get
\be
\frac{dP}{dr} = 
    -\frac{G M \rho}{r^2} \times 
    \frac{ (1 + P/\rho c^2)(1+4\pi r^3 P/m c^2) }
    {1-2 G m /r c^2 }.
\ee
Difference originates from the source of gravity:
in the Newtonian case it is the mass $m$, whereas in the General relativity it is the energy momentum tensor that depend both on the energy density and the pressure.
As a result, energy and pressure give rise to a gravitational fields.

Severness of the GR corrections can be estimated from the so called compactness parameter
\be
x = \frac{GM}{Rc^2} \approx 2.95~M/\Msun \km
\ee


It has an important consequence to the stability of neutron stars:
Successive increase in the pressure to counter the gravity is ultimately self-defeating.

\red{
Solution for a constant density $\rho_0$ gives
\be
P(r) = G \frac{2\pi}{3} \rho_0^2 (R^2 - r^2)
\ee
whereas the GR gives
\be
P(r) = \rho_0^2 c^2 \left[ \frac{ (1-u \left(\frac{r}{R} \right)^2 )^{1/2}
                        - (1-u)^{1/2} }{
                        3(1-u)^{1/2} - (1-u \left( \frac{r}{R} \right)^2 )^{1/2} } \right],
\ee
where $u = 2GM/Rc^2$.
}









