\chapter{Summary of the original publications}
\chapterimage[width=15cm]{wordcloud/chap1b.png}




\section{The author's contribution to the publications} 

\subsection{Paper I.}
Poutanen, J., Nättilä, J., Kajava, J. J. E., Latvala, O.-M., Galloway, D. K., Kuulkers, E., Suleimanov, V. F.: The effect of accretion on the measurement of neutron star mass and radius in the low-mass X-ray binary 4U 1608-52, 2014, MNRAS, 442, 3777–3790,http://dx.doi.org/10.1093/mnras/stu1139

The author of the thesis made contributions to the manuscript, reduced and analyzed the observational X-ray data, and contributed to the scientific discussions related to the manuscript.


\subsection{Paper II.}
Kajava, J. J. E., Nättilä, J, Latvala, O.-M., Pursiainen, M., Poutanen, J., Suleimanov, V. F., Revnivtsev, M. G., Kuulkers, E., Galloway, D. K.: The influence of accretion geometry on the spectral evolution during thermonuclear (type I) X-ray bursts, 2014, MNRAS, 445:4218–4234,http://dx.doi.org/10.1093/mnras/stu2073

The author participated in the reduction and analysis of the observational data, made significant contributions to the development of the data reduction software, and helped in the preparation of the manuscript.


\subsection{Paper III.}
Nättilä, J., Suleimanov, V. F., Kajava, J. J. E., Poutanen, J.: Models of neutron star atmospheres enriched with nuclear burning ashes, 2015, A&A, 581:A83,http://dx.doi.org/10.1051/0004-6361/201526512

The author contributed to the main idea of the paper, independently redesigned the neutron star atmosphere code used for the calculations, and implemented new physical processes to this numerical framework. 
The author also prepared most of the manuscript.


\subsection{Paper IV.}
Nättilä, J., Steiner, A. W., Kajava, J. J. E., Suleimanov, V. F., Poutanen, J.: Equation of state constraints for the cold dense matter inside neutron stars using the cooling tail method, 2016, A&A, 591:A25,http://dx.doi.org/10.1051/0004-6361/201527416

The author independently designed the Bayesian fitting framework for the cooling tail method, reduced and analyzed the X-ray observations, and finally led the equation of state modeling from the observations. 
The author also prepared the manuscript.


\subsection{Paper V.}
Kajava, J. J. E., Nättilä, J., Poutanen, J., Cumming, A., Suleimanov, V., Kuulkers, E.: Detection of burning ashes from thermonuclear X-ray bursts, 2017, MNRAS, 464:L6–L10,http://dx.doi.org/10.1093/mnrasl/slw167

The author contributed to the main idea of this research and was responsible of the atmosphere modeling of the observations. The Bayesian atmosphere spectral model fitting framework was also independently designed by the author. 
Author also made significant contributions to the manuscript.


\subsection{Paper VI.}
Suleimanov, V. F., Poutanen, J., Nättilä, J., Kajava, J. J. E.; Revnivtsev, M. G., Werner, K.: The direct cooling tail method for X-ray burst analysis to constrain neutron star masses and radii, 2017, MNRAS, 466, 906-913,http://dx.doi.org/10.1093/mnras/stw3132

Author helped in designing the fitting framework, based on his own previous results, and contributed to the scientific and statistical discussions of the paper. 
The author also contributed to the manuscript.


\subsection{Paper VII.}
Kuuttila, J., Kajava, J. J. E., Nättilä, J., Motta, S. E., Sanchez-Fernandez, C., Kuulkers, E., Cumming, A., Poutanen, J.: Flux decay during thermonuclear X-ray bursts analysed with the dynamic power-law index method, 2017, A&A, in press,https://arxiv.org/abs/1705.05653

In this paper, the author proposed the usage of the dynamic power-law method and co-supervised the project which was originally based on the Master's thesis of J. Kuuttila. 
Author also made significant contributions to the manuscript.


\subsection{Paper VIII.}
Kajava, J. J. E., Koljonen, K. I. I., Nättilä, J., Suleimanov, V., Poutanen, J.: Variable spreading layer in 4U 1608-52 during thermonuclear X-ray bursts in the soft state, 2017, MNRAS, in press, https://arxiv.org/abs/1707.09479

The author of the thesis took part in the discussion of the theoretical explanation for the obtained observational results and contributed significantly to the statistical analysis of data. 
The author also contributed to the manuscript.


\subsection{Paper IX.}
Nättilä, J. Pihajoki, P.: Radiation from rapidly rotating oblate neutron stars, 2017, A&A, submitted

The author independently proposed the idea of applying the split-Hamilton method to the ray tracing problem of photons, derived the theoretical framework and all the related formulae, designed the numerical code, and prepared most of the manuscript.


\subsection{Paper X.}
Nättilä, J., Miller, M. C., Steiner, A. W., Kajava, J. J. E., Suleimanov, V. F., Poutanen, J.: Atmosphere model fits of thermonuclear X-ray burst cooling tail spectra: new neutron star mass and radius constraints using Bayesian hierarchical modeling, 2017, A&A, submitted

The author independently designed the hierarchical Bayesian fitting framework, implemented it into a code together with M.C. Miller, analyzed the data, and, finally, prepared most of the manuscript together with M.C. Miller.


\subsection{Paper XI.}
Suleimanov, Valery V. F., Kajava, J. J. E., Molkov, S. V., Nättilä, J., Lutovinov, A. A., Werner, K. Poutanen, J.: Basic parameters of the helium accreting X-ray bursting neutron star in 4U 1820-30, 2017, MNRAS, submitted

In this paper, the author took part in the scientific discussion of the results, helped in the statistical analysis and contributed to the manuscript.


