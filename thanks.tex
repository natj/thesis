\chapter*{Acknowledgements}

Looking back on what I actually did during my Ph.D. studies it seems that I applied some kind of a Monte Carlo random algorithm on what I selected as my focus of the year.
On my first year of studies, I learned to program.
On my second year, I learned basic statistics and Bayesian inference.
During my third year, I learned mathematics.
On my last, fourth year I finally grasped some simple astrophysics.
In the last couple of months when starting this thesis project I finally realized what it is about.
During the last days of writing this thesis, I learned not to care too much about non-sense details and just to get it done.
It seems that exactly the opposite order of learning these things would have been optimal.

This has been quite a journey and now when thinking about it, I have no idea how I even ended up here.
When I started in university of Oulu, I had not taken a single course in astronomy.
For some unknown reason I still ended up doing my Bachelor thesis in astrophysics in 2012.
The topic was X-ray bursts from neutron stars.
Now, 5 years later, I am finishing up a Ph.D. thesis on X-ray bursts from neutron stars.
What started as a short bachelor's project turned out to be quite a colorful academic experience for me.
Hence, multiple thanks are also due for many different people.

%After that, I quickly followed by a Master's thesis project on ``Neutron star mass and radius constraints from X-ray bursts''. 
%Four years later I am now finishing up a Ph.D. thesis on almost the same topic.
%After reading that Master's thesis now, it seems that I had no clue about this stuff.
%Maybe same applies to this thesis after 5 years.


#Supervisors:
Juri
Jari
Sergei?

#Defence:
Opponent
Referees

#Other scientific ppl:
Valery
Pauli
Andrew
Cole
Sasha

#group:
Tuomo
Jere
Ilya
Anna
Pavel
Vlad

#HPC:
Pekka M.
Jussi E.
Martti L.
Mikko B.
Sami I.
Sebastian vA.
Seija 

#Observatory:
Robertto
Ronald

#Hese:
Jaakko
Joonas I.
Miku
Vesku
Harri
Olli
Joonas S.
Joonas M.

#perhe:
Ämmi & Eki
Mummi & Ukki

Vilma
Severi

Äiti
Isä

Venla


