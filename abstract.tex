%\begin{abstract}
%Neutron stars are the best.
%\end{abstract}


%When neutron stars (NS) accrete gas from low-mass binary companions, explosive nuclear burning reactions in the NS envelope fuse hydrogen and helium into heavier elements. The resulting thermonuclear (type-I) X-ray bursts produce energy spectra that are fit well with black bodies, but a significant number of burst observations show deviations from Planck spectra.
%
%Neutron star (NS) masses and radii can be estimated from observations of photospheric radius- expansion X-ray bursts, provided the chemical composition of the photosphere, the spectral colour-correction factors in the observed luminosity range, and the emission area during the bursts are known. 
%
%The cooling phase of thermonuclear (type-I) X-ray bursts can be used to constrain neutron star (NS) compactness by comparing the observed cooling tracks of bursts to accurate theoretical atmosphere model calculations. 


%--------------------------------------------------
%intro-intro
%First, a short history of neutron stars is presented, focusing on both theoretical and observational results.
%After this, an introduction to the basic physics of stars is given.
%intro-eos
%Next, the physics of neutron star interiors is discussed in more detail.
%The discussion first focuses on the uppermost layers of the star called an atmosphere.
%This thin layer of plasma is responsible of shaping the emergent radiation.
%Below the atmosphere is a solid layer called the crust.
%In order to understand the physics of the crust, an introduction to the degenerate matter is presented.
%The matter in the crust can then be described by the degenerate electron gas model.
%The crust is, however, only couple of kilometers thick and a bulk of the star consists of a liquid core.
%The EoS of the matter in the core is largely unknown and because of this, we present a way to parameterize the nuclear physics using a so-called polytropic model.
%intro-astro
%The introduction is continued by describing the colorful astrophysics of neutron stars.
%We mainly focus on presenting the basics of accretion, an astrophysical process where matter is transferred into a compact object such as a neutron star.
%intro-constraints
%Lastly, we show how the neutron star size can be inferred from the X-ray burst observations.



\vspace{2.0cm}
\chapter*{Abstract}

%background
Neutron stars are one of the most dense objects in the Universe.
However, the exact description of the equation of state (EoS) of the cold ultra-dense matter inside them is still a mystery.
In this thesis, we measure the size of the neutron stars using astrophysical observations of so-called X-ray bursts that are produced by thermonuclear runaways in the uppermost layers of the star.
By measuring the size, we can then ultimately set constraints on the nuclear physics of the unknown interiors.

The size measurements are done by comparing the cooling of the neutron star surfaces after the bursts to the theoretical atmosphere model calculations.
Hence, accurate modeling of the emergent radiation from the atmospheres is needed.
In the first part of this thesis, I have studied how the emergent spectra differ if the atmosphere is enriched with nuclear burning ashes from the bursts.
This gives us new tools to understand and interpret the X-ray burst observations.
In addition, I have shown how to take the rapid rotation into account when describing the radiation from oblate neutron stars.

Together with accurate models, we must also be careful in selecting only those bursts that are not actively heated by the infalling material.
In the second part of the thesis, I have focused on studying the astrophysical environments of the X-ray bursts in order to measure the effect of accretion on the mass and radius determinations.
Importantly, it is shown that only the bursts that occur during the low accretion rate state can be used for the mass and radius determination because otherwise the accretion flow might influence the cooling of the star's surface.

After taking these steps into account, it is possible to set constraints on the mass, radius, distance, and atmosphere composition of neutron stars exhibiting X-ray bursts.
In the third part, I have used this knowledge to constrain the mass and radius of neutron stars using the low accretion rate X-ray bursts.
More specifically, I have constrained the radius of xx to be yyy.

%method results
%conclusions

These results are in a good agreement with the nuclear physical predictions and open up a new way of connecting the unknown nuclear physics to astrophysics.






\vspace{2.0cm}
\chapter*{Tiivistelmä}
Neutronitähdet on parhaita.
